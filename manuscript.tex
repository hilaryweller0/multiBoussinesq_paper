%% LyX 2.2.4 created this file.  For more info, see http://www.lyx.org/.
%% Do not edit unless you really know what you are doing.
\documentclass[12pt,times]{extarticle}
\usepackage{mathptmx}
\usepackage[T1]{fontenc}
\usepackage[latin9]{inputenc}
\usepackage{geometry}
\geometry{verbose,tmargin=2.5cm,bmargin=2.5cm,lmargin=2.5cm,rmargin=2.5cm}
\usepackage{array}
\usepackage{rotating}
\usepackage{amsmath}
\usepackage{amssymb}
\usepackage{graphicx}
\usepackage[authoryear]{natbib}

\makeatletter

%%%%%%%%%%%%%%%%%%%%%%%%%%%%%% LyX specific LaTeX commands.
%% Because html converters don't know tabularnewline
\providecommand{\tabularnewline}{\\}

%%%%%%%%%%%%%%%%%%%%%%%%%%%%%% User specified LaTeX commands.
\newcommand{\nicefrac}[2]{\ensuremath ^{#1}\!\!/\!_{#2}}
\usepackage { fancybox}
\usepackage[export]{adjustbox}

\usepackage{todonotes}
%\usepackage{afterpage}

%\usepackage[switch]{lineno}

\usepackage[colorlinks,bookmarksopen,bookmarksnumbered,citecolor=red,urlcolor=red]{hyperref}

\newcommand\BibTeX{{\rmfamily B\kern-.05em \textsc{i\kern-.025em b}\kern-.08em
T\kern-.1667em\lower.7ex\hbox{E}\kern-.125emX}}

%\usepackage{moreverb}

\usepackage{flushend}

\makeatother

\begin{document}

\title{Multi-fluids for Representing Sub-filter-scale Buoyant Convection}

\author{Hilary Weller, William McIntyre, Daniel Shipley and Peter Clark}
\maketitle
\begin{abstract}
The important bits:

1. entrainment is equal to $-\nabla\cdot\mathbf{u}_{i}$

2. the difference in pressure between fluids is $-\gamma\nabla\cdot\mathbf{u}_{i}$

3. Two 2D dry test cases, rising buoyant bubble and radiative-convective
equilibrium.

4. Multi-fluid model in 1D reproduces conditionally averaged resolved
2D simulations
\end{abstract}

\section{Introduction}

\section{The Conditionally Averaged Boussinesq Equations}

Many of the challenges of representing convection with the conditionally
averaged Euler equations \citep[eg those described in][]{WM19} carry
over to the Boussinesq equations. Therefore for simplicity we focus
on the conditionally averaged Boussinesq equations:
\begin{eqnarray}
\frac{\partial\sigma_{i}}{\partial t}+\nabla\cdot\sigma_{i}\mathbf{u}_{i} & = & \sum_{j\ne i}\sigma_{j}S_{ji}-\sigma_{i}S_{ij}\label{eq:sigma}\\
\frac{D_{i}\mathbf{u}_{i}}{Dt}+\nabla P_{i} & = & b_{i}\mathbf{k}+\nu\nabla^{2}\mathbf{u}_{i}+\sum_{j\ne i}\frac{\sigma_{j}}{\sigma_{i}}S_{ji}\left(\mathbf{u}_{ji}^{T}-\mathbf{u}_{i}\right)-S_{ij}(\mathbf{u}_{ij}^{T}-\mathbf{u}_{i})-\mathbf{D}_{ij}\label{eq:mom}\\
\frac{D_{i}b_{i}}{Dt} & = & \alpha\nabla^{2}b_{i}+\sum_{j\ne i}\frac{\sigma_{j}}{\sigma_{i}}S_{ji}\left(b_{ji}^{T}-b_{i}\right)-S_{ij}\left(b_{ij}^{T}-b_{i}\right)\label{eq:b}\\
\sum_{i}\nabla\cdot\sigma_{i}\mathbf{u}_{i} & = & \nabla\cdot\sum_{i}\sigma_{i}\mathbf{u}_{i}=0\label{eq:divFree}\\
\sum_{i}\sigma_{i} & = & 1\label{eq:sumOne}
\end{eqnarray}
given the definitions of variables in table \ref{tab:defns}. 

\begin{table}[h]
\begin{tabular}{c>{\raggedright}p{0.85\textwidth}}
$\sigma_{i}$ & Volume fraction of fluid $i$ so that $\sum_{i}\sigma_{i}=1$\tabularnewline
$\mathbf{u}_{i}$ & Velocity of fluid $i$ ($\text{m}\text{s}^{-1}$)\tabularnewline
$w_{i}$ & Vertical component of $\mathbf{u}_{i}$ ($\text{m}\text{s}^{-1}$)\tabularnewline
$S_{ij}$ & Transfer from fluid $i$ to $j$ ($\text{s}^{-1}$)\tabularnewline
$P_{i}$ & Non-hydrostatic pressure of fluid $i$ $=p^{\prime}/\rho_{r}$ ($\text{m}^{2}\text{s}^{-2}$)\tabularnewline
$\psi_{ij}^{T}$ & Value of variable $\psi$ transferred from fluid $i$ to $j$\tabularnewline
$b$ & Buoyancy defined as -$g\rho^{\prime}/\text{\ensuremath{\rho}}_{r}$
where $\rho^{\prime}$ are departures in density from a horizontally
uniform reference, $\rho_{r}$ ($\text{m}\text{s}^{-2}$)\tabularnewline
$b_{i}$ & Buoyancy of fluid $i$ ($\text{m}\text{s}^{-2}$)\tabularnewline
$D_{i}\big/Dt$ & Total derivative with respect to fluid $i$ $=\partial/\partial t+\mathbf{u}_{i}\cdot\nabla$\tabularnewline
$\mathbf{D}_{ij}$ & Drag on fluid $i$ from fluid $j$ ($\text{m}\text{s}^{-2}$)\tabularnewline
$\gamma$ & Coefficient for setting the pressure local for each fluid ($\text{m}^{2}\text{s}^{-1}$)\tabularnewline
$C_{D}$ & Drag coefficient $=0$ or 1\tabularnewline
$r_{c}$ & Plume radius used for defining the drag between fluids (m)\tabularnewline
$\alpha$ & Diffusivity of buoyancy ($\text{m}^{2}\text{s}^{-1}$)\tabularnewline
$\nu$ & Viscosity ($\text{m}^{2}\text{s}^{-1}$)\tabularnewline
\end{tabular}

\caption{Definitions of variables for the conditionally average Boussinesq
equations.\label{tab:defns}}
\end{table}


\subsection{Pressure of each fluid}

\citet{TEB19} and \citet{WM19} assumed that each co-located fluid
shared the same pressure, leading to unstable equations that \citet{TEB19}
stabilised using diffusion of vertical velocity and \citet{WM19}
stabilised using mass exchanges or drag between the fluids. \citet{TEB19}
achieved realistic results in single column experiments of convective
boundary layer but the amount of diffusion needed would spoil simulations
in the free troposphere. The stabilisation used by \citet{WM19} meant
that the two fluids tended to move as one which defeats the purpose
of using multi-fluid modelling. 

The instabilities reported by \citet{WM19} showed growing divergence
in a fluid with vanishing volume fraction. It is mass convergence
rather than velocity convergence that leads to pressure anomalies
that remove the convergence. Therefore if the volume (or mass) fraction
of fluid is vanishingly low then convergence cannot lead to an anomaly
in the total pressure.

We will assume that pressure in each fluid is given by a parameterised
perturbation from the total pressure, $P$:
\begin{equation}
P_{i}=P+p_{i}.
\end{equation}

\citet{BN86} proposed a parameterisation for pressure differences
in a multi-phase explosion model that has been widely used in mathematics
and engineering for simulating fluidised beds of granular material
\citep[eg.][]{EHM92} and two compressible fluids \citep[eg.][]{SA99}.
In these methods, advection equations are written for the volume fraction,
$\sigma_{i}$, and conservation equations are written for mass fraction,
$\sigma_{i}\rho_{i}$. In our notation, ignoring transfers between
fluids, these equations would be:
\begin{eqnarray}
\frac{\partial\sigma_{i}}{\partial t}+\mathbf{V}\cdot\nabla\sigma_{i} & = & \mu\left(P_{i}-P_{j}\right)\label{eq:BNcompaction}\\
\frac{\partial\sigma_{i}\rho_{i}}{\partial t}+\nabla\cdot\left(\sigma_{i}\rho_{i}\mathbf{u}_{i}\right) & = & 0\label{eq:rhoTransport}
\end{eqnarray}
where $\mathbf{V}$ is the velocity of the interface between fluids,
$\mu$ is the compaction viscosity which controls how quickly the
pressure of each fluid relaxes to the mean pressure and index $j$
refers to the other fluid in a two fluid system. In our Boussinesq
system $\rho_{i}$ does not appear in the continuity equation. Different
authors make different assumptions about the interface velocity but
if we assume that $\mathbf{V}=\sum_{i}\sigma_{i}\mathbf{u}_{i}$ and
$\rho_{i}$ is constant then eqns (\ref{eq:divFree}-\ref{eq:rhoTransport})
imply that $\mu p_{i}=-\sigma_{i}^{2}\nabla\cdot\mathbf{u}_{i}$.
If we assume that $\mathbf{V}=\mathbf{u}_{i}$ for each fluid then
we get $\mu p_{i}=-\sigma_{i}\nabla\cdot\mathbf{u}_{i}$.

\citet{TEB19} stabilised the multi-fluid equations by adding a diffusion
term $\nu\partial w^{2}/\partial z^{2}$ to the RHS of the $w$ equation.
If a fluid perturbation pressure is set to $-\nu\partial w/\partial z$
this is equivalent to the diffusion term in one dimension. However
addition a diffusion term will harm three dimensional simulations
of resolved convection in the free troposphere whereas adding a fluid
perturbation pressure will not. Inspired by both \citet{BN86} and
\citet{TEB19} we assume that the pressure in each fluid is given
by:
\begin{equation}
p_{i}=-\gamma\nabla\cdot\mathbf{u}_{i}\label{eq:Pi_div}
\end{equation}
where $\gamma$ is a free coefficient. This parameterisation and sensitivity
to $\gamma$ will be evaluated in section \ref{sec:results}.

\subsection{Entrainment and Detrainment}

We consider dry air without turbulence so use the dynamic entrainment
described by \citet{HC51,AK67,DBF+13}:
\begin{equation}
S_{ij}=\begin{cases}
-\nabla\cdot\mathbf{u}_{i} & \text{if }\nabla\cdot\mathbf{u}_{i}<0\\
0 & \text{otherwise}
\end{cases}
\end{equation}
which is necessary to stop air in one fluid from accumulating when
vertical motion ceases (at the top of a rising plume or at the bottom
of descending air). We will also test a common form of lateral entrainment
\citep{DBF+13}:
\begin{equation}
\varepsilon=\frac{0.2}{r_{c}}\implies S_{01}=w_{1}\frac{0.2}{r_{c}}
\end{equation}
where $\varepsilon$ is the well know fractional entrainment rate
and $r_{c}$ is the cloud or plume radius. 

\subsection{Drag in the Momentum Equation\label{subsec:drag}}

Pressure differences between the fluids can lead to form drag which
is parameterised following \citet{SW69,RC15,WM19} as: 
\begin{equation}
\mathbf{D}_{ij}=\frac{\sigma_{j}C_{D}}{r_{c}}|\mathbf{u}_{i}-\mathbf{u}_{j}|\left(\mathbf{u}_{i}-\mathbf{u}_{j}\right)\label{eq:dragBubble}
\end{equation}
where $C_{D}$ is a drag coefficient, $r_{c}$ is a cloud radius.
Sensitivity to $C_{D}/r_{c}$is explored in section \ref{sec:results}. 

\subsection{The Buoyancy and the Momentum of the Mass that is Transferred}

The different fluids may not be well mixed so the fluid transferred
may not have the mean properties of the fluid it is leaving. In fact,
the most buoyant air should be transferred from fluid 0 to fluid 1
and vice versa and the air with least downward momentum is transferred
from 0 to 1 and vice versa. The properties of the fluids transferred
should depend on modelling of sub-grid-scale variability which is
beyond the scope of this paper. Conditional averaging of high resolution
single fluid solutions depends on the sign of $w$ and so it is sensible
to assume that the vertical velocity of the fluid transferred is zero.
For two dimensional multi-fluid simulations we assume that the horizontal
velocity transferred is equal to the mean horizontal velocity of the
fluid transferred from. Therefore
\begin{equation}
\mathbf{u}_{ij}^{T}=\begin{pmatrix}u_{i}\\
v_{i}\\
0
\end{pmatrix}.
\end{equation}
For buoyancy, following \citet{TEB19}, we use:
\begin{equation}
b_{ij}^{T}=\theta_{b}b_{i}+(1-\theta_{b})b_{j}
\end{equation}
and present results using $\theta_{b}=\frac{1}{2}$ and $\theta_{b}=1$.
Note that $\theta_{b}\ne1$ implies that the buoyancy of the fluid
transferred depends on the properties of the receiving fluid which
is not logical and can lead to unbounded values of $b_{i}$ in the
fluid that is loosing mass.

\subsection{Dimensional Analysis}

Dimensional analysis can guide us in our choice of $\gamma$ for the
parameterisation of the pressure difference between fluids. We make
the following assumptions:
\begin{enumerate}
\item The flow is buoyancy dominated.
\item The flow is close to inviscid.
\item The flow is slowly varying.
\item Fluid $i$ properties are anomalies from a neutrally stable resting
mean state.
\item We consider only the vertical direction where $w_{ij}^{T}=0$.
\item $P_{i}=P-\gamma\nabla\cdot\mathbf{u}_{i}\approx-\gamma\nabla\cdot\mathbf{u}_{i}=-\gamma\frac{\partial w_{i}}{\partial z}$
\item There are two fluids, $i$ and $j$
\item Without loss of generality we assume that $S_{ij}=-\frac{\partial w_{i}}{\partial z}>0$
and $S_{ji}=0$.
\end{enumerate}
This leads to the following balance in the momentum equation:
\begin{equation}
\frac{\partial w_{i}^{2}}{\partial z}-\gamma\frac{\partial^{2}w_{i}}{\partial z^{2}}=b_{i}\label{eq:wi_balances}
\end{equation}
If we also assume that fluid $i$ properties are anomalies from a
resting mean state. We non-dimensionalise using a length scale $L^{*}$,
a buoyancy scale $B^{*}$ and a time scale $T^{*}=(L^{*}/B^{*})^{\frac{1}{2}}$
to get the non-dimensional variables:
\begin{eqnarray*}
\tilde{w} & = & w_{i}(L^{*}B^{*})^{-\frac{1}{2}}\\
\tilde{b} & = & b_{i}/B^{*}\\
\tilde{z} & = & z/L^{*}.
\end{eqnarray*}
Then the non-dimensional version of (\ref{eq:wi_balances}) is then:
\begin{equation}
\underbrace{{\frac{\partial\tilde{w}^{2}}{\partial\tilde{z}}}}_{\text{advection}}-\underbrace{{\frac{\gamma}{B^{*\frac{1}{2}}L^{*\frac{3}{2}}}\frac{\partial^{2}\tilde{w}}{\partial\tilde{z}^{2}}}}_{\text{stabilisation}}=\underbrace{\tilde{b}}_{\text{buoyancy}}.\label{eq:wi_nonDom}
\end{equation}
We have assumed that the flow is buoyancy dominated so $\tilde{b}$
is $O(1)$. The stabilisation term must be large enough to smooth
out oscillations in $\tilde{w}$ due to advection and buoyancy but
not too large to remove all variability in $\tilde{w}$ so we need:
\begin{equation}
\gamma\sim B^{*\frac{1}{2}}L^{*\frac{3}{2}}.
\end{equation}


\section{Numerical Solution}

The spatial discretisation follows exactly \citet{WM19}, solving
advective form equations using monotonic, finite volume advection
and C-grid, Lorenz staggering. The time-stepping is Crank-Nicolson
with no off-centering and with deferred correction of explicitly solved
variables. It differs from \citet{WM19} because here we solve the
Boussinesq equations and we need stable solutions for equations with
a separate pressure in each fluid. We use two outer iterations per
time step to update explicitly solved variables. For the first outer
iteration, predicted values for time $n+1$ are set to those from
time level $n$ and are given the superscript $\ell$. For the second
iteration, values at $\ell$ are set to those from the end of the
first iteration. The implicit numerical method for applying the transfers
between fluids is specific for two fluids because it involves the
inversion of $2\times2$ matrices. 

The prognostic variables are the $b_{i}$ in cell centres, the $\sigma_{i}$
is cell centres and the velocity flux: the velocity at cell faces
dotted with the area vector for each face (normal to the face), $u_{i}=\mathbf{u}_{i}\cdot\mathbf{S}$.
The pressures, $P$ and $p_{i}$ are diagnostic.

The first equation to be solved each outer iteration is eqn (\ref{eq:sigma})
for each $\sigma_{i}$, operator split; first applying advection,
then correcting $\sum_{i}\sigma_{i}=1$ and then applying mass transfers:
\begin{eqnarray}
\sigma_{i}^{\prime} & = & \sigma_{i}^{n}-\frac{\Delta t}{2}\nabla\cdot\left\{ \left(\mathbf{u}_{i}^{n}+\mathbf{u}_{i}^{\ell}\right)\left(\sigma_{i}^{n}\right)_{vL}\right\} \\
\sigma_{i}^{\prime\prime} & = & \sigma_{i}^{\prime}\bigg/\sum_{i}\sigma_{i}^{\prime}\\
S_{ij} & = & \begin{cases}
-\nabla\cdot\mathbf{u}_{i}^{\ell} & \text{if }\nabla\cdot\mathbf{u}_{i}^{\ell}<0\\
0 & \text{otherwise}
\end{cases}\\
\sigma_{i}^{n+1} & = & \sigma_{i}^{\prime\prime}+\Delta t\sum_{j\ne i}\sigma_{j}^{\prime\prime}S_{ji}-\sigma_{i}^{\prime\prime}S_{ij}
\end{eqnarray}
where $\Delta t$ is the time step and the superscripts $\prime$
and $\prime\prime$ denote intermediate values. $\sigma_{i}^{n}$
is interpolated from cell centres onto cell faces using monotonic
van-Leer advection (operator $(\sigma_{i}^{n})_{vL}$) as in \citet{WM19}
using exclusively values at time level $n$ for best accuracy and
guaranteed monotonicity. For calculating the divergence, the normal
component of $\mathbf{u}_{i}$ is needed at cell faces which are prognostic
variables of the C-grid. The $S_{ij}$ are limited to ensure $0<\sigma_{i}<1\ \forall i$. 

Next eqn (\ref{eq:b}) is solved for the buoyancy in each fluid. The
transfer terms can be very large due to the presence of $\frac{\sigma_{j}}{\sigma_{i}}S_{ij}$
in the transfer term for $b_{i}$. 

so they are treated operator split and implicitly, with $b_{0}^{n+1}$
coming from the solution of a $2\times2$ matrix and $b_{1}^{n+1}$
being the back-substitution:
\begin{eqnarray}
b_{i}^{\prime} & = & b_{i}^{n}+\frac{\Delta t}{2}\left(b_{it}^{n}+b_{it}^{\ell}\right)\\
\text{where }b_{it} & = & -\nabla\cdot\left(\mathbf{u}_{i}(b_{i})_{vL}\right)+b_{i}\nabla\cdot\mathbf{u}_{i}-N^{2}w_{i}+\alpha\nabla^{2}b_{i}\\
b_{0}^{n+1} & = & \frac{\left(1+\Delta t\left(\frac{\sigma_{0}}{\sigma_{1}}\right)^{\prime\prime}S_{01}\right)b_{0}^{\prime}+\Delta t\left(\frac{\sigma_{1}}{\sigma_{0}}\right)^{\prime\prime}S_{10}b_{1}^{\prime}}{1+\Delta t\left(\frac{\sigma_{0}}{\sigma_{1}}\right)^{\prime\prime}S_{01}+\Delta t\left(\frac{\sigma_{1}}{\sigma_{0}}\right)^{\prime\prime}S_{10}}\\
b_{1}^{n+1} & = & \frac{\Delta t\left(\frac{\sigma_{0}}{\sigma_{1}}\right)^{\prime\prime}S_{01}b_{0}^{n+1}+b_{1}^{\prime}}{1+\Delta t\left(\frac{\sigma_{0}}{\sigma_{1}}\right)^{\prime\prime}S_{01}}.
\end{eqnarray}
The use of $(\sigma_{0}/\sigma_{1})^{\prime\prime}$ before the mass
transfers is necessary to ensure bounded and conservative transfers,
as described by \cite{MWH1x}.

Next a Poisson equation is solved to calculate the total pressure
to ensure that the total velocity is divergence free. The pressure
equation also provides updates for the velocity flux, $u_{i}$. First
an intermediate velocity flux is calculated with partial updates:
\begin{eqnarray}
u_{i}^{\prime}= & u_{i}^{n}+\frac{\Delta t}{2} & \biggl(\left(-\nabla\cdot(\mathbf{u}_{i}\mathbf{u}_{i})+\mathbf{u}_{i}\nabla\cdot\mathbf{u}_{i}+b_{i}\mathbf{k}\right)_{f}^{\ell}\cdot\mathbf{S}-\nabla_{f}p_{i}^{\ell}\\
 &  & +\left(-\nabla\cdot(\mathbf{u}_{i}\mathbf{u}_{i})+\mathbf{u}_{i}\nabla\cdot\mathbf{u}_{i}+b_{i}\mathbf{k}\right)_{f}^{n}\cdot\mathbf{S}-\nabla_{f}(P+p_{i})^{n}\biggr)
\end{eqnarray}
where operator $()_{f}$ means linear interpolation from cell centres
onto cell faces and $\nabla_{f}P$ is a compact discretisation of
$(\nabla P)_{f}\cdot\mathbf{S}$: ie the normal component of the pressure
gradient calculated from just the values of pressure either side of
the face. Once we have calculated $P^{n+1}$, the velocity flux (without
momentum transfers) is updated from the back-substitution:
\begin{equation}
u_{i}^{\prime\prime}=u_{i}^{\prime}-\frac{\Delta t}{2}\nabla_{f}P^{n+1}.\label{eq:backSub}
\end{equation}
The Poisson equation for $P$ is found by multiplying each eqn (\ref{eq:backSub})
by $\sigma_{i}$, summing over all fluids and taking the divergence.
Knowing that $\nabla\cdot\sum_{i}\sigma_{i}\mathbf{u}_{i}=0$, $\sum_{i}\sigma_{i}=1$
and $\sum_{i}\sigma_{i}p_{i}=0$ gives:
\begin{equation}
\nabla\cdot\sum_{i}\sigma_{i}u_{i}^{\prime}-\frac{\Delta t}{2}\nabla^{2}P^{n+1}=0
\end{equation}
which is solved to find $P^{n+1}$. 

Next, the momentum transfers due to mass transfer and due to drag
are calculated implicitly, similar to the implicit transfers of buoyancy,
calculating $u_{i}^{n+1}$ from $u_{i}^{\prime\prime}$. These do
not influence the total $P$ or total divergence because momentum
is transferred conservatively. However they do influence $p_{i}$
which is why they are calculated before $p_{i}$. The final stage
of the time step iteration is to solve Helmholtz equations for each
$p_{i}$:
\begin{equation}
\sigma_{i}^{n+1}p_{i}^{n+1}=-\gamma\nabla\cdot\left(\sigma_{if}^{n+1}u_{i}^{n+1}-\frac{\Delta t}{2}\sigma_{if}^{n+1}\nabla_{f}\left(p_{i}^{n+1}-p_{i}^{\ell}\right)\right).\label{eq:pi_Helm}
\end{equation}
Both $p_{i}^{n+1}$ and $p_{i}^{\ell}$ appear in eqn (\ref{eq:pi_Helm})
because $u_{i}^{n+1}$ already contains contributions from $p_{i}^{\ell}$
so these are removed before updating $p_{i}^{n+1}$. No back substitution
for $u_{i}$ is done after calculating $p_{i}^{n+1}$ because this
would destroy the divergence free constraint. Therefore $p_{i}^{n+1}$
only satisfies eqn (\ref{eq:Pi_div}) exactly if the outer iterations
converge and $p_{i}^{\ell}=p_{i}^{n+1}$. 

The final stage of the time step iteration is to reconstruct $\mathbf{u}_{i}$
from $u_{i}$ following \citet{WM19}. The stages calculating $P$,
$p_{i}$ and $\mathbf{u}_{i}$ are repeated twice per time step iteration.
The Poisson and Helmholtz equations are solved to a tolerance of $10^{-6}$.

\section{Validation Test Cases\label{sec:results}}

Two test cases are developed to tune parameters and to validate the
use of multi-fluid equations to represent dry, two-dimensional sub-grid-scale
convection. The first is steady state with heat transfer at ground
and radiative cooling. The second is transient; the standard rising
bubble of \citet{BF02}. We aim to reproduce the vertical heat transport,
mean velocity of the rising air, bouyancy of the rising air and pressure
differences between two fluids in a single column two fluid model.
For both test cases the reference solution is a single fluid solution
of the two-dimensional Boussinesq equations at high horizontal and
vertical resolution. 

\subsection{Rising Bubble}

The test case of \citet{BF02} was designed for fully compressible
models but similar solutions are obtained solving Boussinesq equations.
For the Boussinesq version we set $N^{2}=0$ in a domain of height
10\,km and width 20\,km initially at rest. The buoyant perturbation
is given by:
\begin{equation}
b=\frac{1}{15}\cos^{2}\frac{\pi L}{2}\label{eq:thetaPerturb}
\end{equation}
for $L<1$ where $L=\sqrt{\left(\frac{x-x_{c}}{x_{r}}\right)^{2}+\left(\frac{z-z_{c}}{z_{r}}\right)^{2}}$,
$x_{c}=10\ \text{km}$, $z_{c}=2\ \text{km}$ and $x_{r}=z_{r}=2\ \text{km}$.
100\,m grid spacing is used in the $x$ and $z$ directions for the
two dimensional reference simulation and $\Delta z=100\ \text{m}$
is used in the vertical for the single column simulations. All simulations
use a time-step of 2\,s.

\subsubsection{Single Fluid Solutions}

The solution of the rising buoyant bubble solving the Boussinesq equations
for 1000s is shown in \ref{fig:bubble} with buoyancy coloured and
pressure contoured. The solution is very similar to the fully compressible
solution \citep{BF02}. The pressure gradients will accelerate the
flow beneath the bubble, decelerate the flow above the bubble and
pull in air horizontally. Horizontally and conditionally averaged
fields, conditioned on $w$ are shown in the top row of figure \ref{fig:bubble_singleUnderRes}.
$\sigma_{1}$ is the volume fraction at every height that is rising,
$b_{0}$ and $b_{1}$ are the buoyancy conditioned on $w$, $w_{0}$
and $w_{1}$ are the horizontal averaging rising and falling vertical
velocity and $P_{0}$ and $P_{1}$ are the pressure of the rising
and falling fluid. An accurate multi-fluid model of convection should
reproduce these fields approximately at coarser horizontal resolution
or in a single column model. 

\begin{figure}
\noindent \begin{centering}
\includegraphics[width=1\textwidth]{/home/hilary/OpenFOAM/hilary-dev/run/hilary/warmBubble/Boussinesq/singleFluid/resolved/1000/bP}
\par\end{centering}
\noindent \begin{centering}
\includegraphics[width=0.75\textwidth]{/home/hilary/OpenFOAM/hilary-dev/run/hilary/warmBubble/Boussinesq/singleFluid/resolved/legends/bP_b}
\par\end{centering}
\caption{The single fluid, resolved rising bubble after 1000s. Buoyancy is
coloured and pressure is contoured, every $10\text{m}^{2}\text{s}^{-2}$,
negative contours dashed. \label{fig:bubble}}
\end{figure}

\citet{TEB19} and \citet{WM19} assumed that both fluids would share
the same pressure with differences parameterised as drag. The pressure
in figure \ref{fig:bubble} shows that this is a bad assumption. Drag
would always act to slow down the fluids whereas the negative pressure
gradients in the rising fluid just below the bubble are acting to
accelerate the rising motion. 

When resolution is insufficient to resolve the rising bubble, the
vertical transports are too weak, motivating the need for convection
parameterisation. The horizontal averages are shown in figure \ref{fig:bubble_singleUnderRes}
for horizontal resolutions of nine columns, five columns, three columns
and a single column all for the 20km wide domain and all still using
$\Delta z=100\ \text{m}$. These simulations are initialised by conservative
horizontal averaging from the full resolution solution. As the horizontal
resolution is reduced, the vertical transport reduces until there
is no movement at all for a single column: there is no possibility
of fluid rising because in a single column with a single fluid there
can be no compensating subsidence. 

\begin{figure}
\noindent %
\begin{tabular}{ccccc}
\begin{turn}{90}
200 Columns
\end{turn} & \includegraphics[scale=0.6]{/home/hilary/OpenFOAM/hilary-dev/run/hilary/warmBubble/Boussinesq/singleFluid/resolved/hMean/1000/sigma} & \includegraphics[scale=0.6]{/home/hilary/OpenFOAM/hilary-dev/run/hilary/warmBubble/Boussinesq/singleFluid/resolved/hMean/1000/b} & \includegraphics[scale=0.6]{/home/hilary/OpenFOAM/hilary-dev/run/hilary/warmBubble/Boussinesq/singleFluid/resolved/hMean/1000/w} & \includegraphics[scale=0.6]{/home/hilary/OpenFOAM/hilary-dev/run/hilary/warmBubble/Boussinesq/singleFluid/resolved/hMean/1000/P}\tabularnewline
\begin{turn}{90}
9 Columns
\end{turn} & \includegraphics[scale=0.6]{/home/hilary/OpenFOAM/hilary-dev/run/hilary/warmBubble/Boussinesq/singleFluid/nineColumn/hMean/1000/sigma} & \includegraphics[scale=0.6]{/home/hilary/OpenFOAM/hilary-dev/run/hilary/warmBubble/Boussinesq/singleFluid/nineColumn/hMean/1000/b} & \includegraphics[scale=0.6]{/home/hilary/OpenFOAM/hilary-dev/run/hilary/warmBubble/Boussinesq/singleFluid/nineColumn/hMean/1000/w} & \includegraphics[scale=0.6]{/home/hilary/OpenFOAM/hilary-dev/run/hilary/warmBubble/Boussinesq/singleFluid/nineColumn/hMean/1000/P}\tabularnewline
\begin{turn}{90}
5 Columns
\end{turn} & \includegraphics[scale=0.6]{/home/hilary/OpenFOAM/hilary-dev/run/hilary/warmBubble/Boussinesq/singleFluid/fiveColumn/hMean/1000/sigma} & \includegraphics[scale=0.6]{/home/hilary/OpenFOAM/hilary-dev/run/hilary/warmBubble/Boussinesq/singleFluid/fiveColumn/hMean/1000/b} & \includegraphics[scale=0.6]{/home/hilary/OpenFOAM/hilary-dev/run/hilary/warmBubble/Boussinesq/singleFluid/fiveColumn/hMean/1000/w} & \includegraphics[scale=0.6]{/home/hilary/OpenFOAM/hilary-dev/run/hilary/warmBubble/Boussinesq/singleFluid/fiveColumn/hMean/1000/P}\tabularnewline
\begin{turn}{90}
3 Columns
\end{turn} & \includegraphics[scale=0.6]{/home/hilary/OpenFOAM/hilary-dev/run/hilary/warmBubble/Boussinesq/singleFluid/threeColumn/hMean/1000/sigma} & \includegraphics[scale=0.6]{/home/hilary/OpenFOAM/hilary-dev/run/hilary/warmBubble/Boussinesq/singleFluid/threeColumn/hMean/1000/b} & \includegraphics[scale=0.6]{/home/hilary/OpenFOAM/hilary-dev/run/hilary/warmBubble/Boussinesq/singleFluid/threeColumn/hMean/1000/w} & \includegraphics[scale=0.6]{/home/hilary/OpenFOAM/hilary-dev/run/hilary/warmBubble/Boussinesq/singleFluid/threeColumn/hMean/1000/P}\tabularnewline
\begin{turn}{90}
1 Column
\end{turn} & \includegraphics[scale=0.6]{/home/hilary/OpenFOAM/hilary-dev/run/hilary/warmBubble/Boussinesq/singleFluid/singleColumn/hMean/1000/sigma} & \includegraphics[scale=0.6]{/home/hilary/OpenFOAM/hilary-dev/run/hilary/warmBubble/Boussinesq/singleFluid/singleColumn/hMean/1000/b} & \includegraphics[scale=0.6]{/home/hilary/OpenFOAM/hilary-dev/run/hilary/warmBubble/Boussinesq/singleFluid/singleColumn/hMean/1000/w} & \includegraphics[scale=0.6]{/home/hilary/OpenFOAM/hilary-dev/run/hilary/warmBubble/Boussinesq/singleFluid/singleColumn/hMean/1000/P}\tabularnewline
\end{tabular}

\caption{The rising bubble after 1000s simulated with coarse horizontal resolution
and a single fluid. Horizontal averages are conditioned based on $w$.
\label{fig:bubble_singleUnderRes}}
\end{figure}


\subsubsection{Multi-fluid Solutions}

\begin{figure}
\begin{tabular}{cccc}
$\sigma_{1}$ & $b$, $b_{0}$ and $b_{1}$ & $w_{0}$ and $w_{1}$ & $P_{0}$ and $P_{1}$\tabularnewline
\multicolumn{4}{c}{Results at $t=1000\ $s using $S_{ij}=0$, $p_{i}=0$, $C_{D}/r_{c}=0$}\tabularnewline
\includegraphics[scale=0.6]{/home/hilary/OpenFOAM/hilary-dev/run/hilary/warmBubble/Boussinesq/multiFluid/singleColumn_noTransfer/1000/sigmaCompare} & \includegraphics[scale=0.6]{/home/hilary/OpenFOAM/hilary-dev/run/hilary/warmBubble/Boussinesq/multiFluid/singleColumn_noTransfer/1000/bCompare} & \includegraphics[scale=0.6]{/home/hilary/OpenFOAM/hilary-dev/run/hilary/warmBubble/Boussinesq/multiFluid/singleColumn_noTransfer/1000/wCompare} & \includegraphics[scale=0.6]{/home/hilary/OpenFOAM/hilary-dev/run/hilary/warmBubble/Boussinesq/multiFluid/singleColumn_noTransfer/1000/Pcompare}\tabularnewline
\multicolumn{4}{c}{Results at $t=1000\ \text{s}$ using $S_{ij}=0$, $p_{i}=0$, $C_{D}/r_{c}=1/2000\text{m}^{-1}$}\tabularnewline
\includegraphics[scale=0.6]{/home/hilary/OpenFOAM/hilary-dev/run/hilary/warmBubble/Boussinesq/multiFluid/singleColumn_noTransfer_drag/1000/sigmaCompare} & \includegraphics[scale=0.6]{/home/hilary/OpenFOAM/hilary-dev/run/hilary/warmBubble/Boussinesq/multiFluid/singleColumn_noTransfer_drag/1000/bCompare} & \includegraphics[scale=0.6]{/home/hilary/OpenFOAM/hilary-dev/run/hilary/warmBubble/Boussinesq/multiFluid/singleColumn_noTransfer_drag/1000/wCompare} & \includegraphics[scale=0.6]{/home/hilary/OpenFOAM/hilary-dev/run/hilary/warmBubble/Boussinesq/multiFluid/singleColumn_noTransfer_drag/1000/Pcompare}\tabularnewline
\multicolumn{4}{c}{Results at $t=1000\ \text{s}$ using $S_{ij}=0$, $p_{i}=-10^{4}\nabla\cdot\mathbf{u}_{i}$,
$C_{D}/r_{c}=0$}\tabularnewline
\includegraphics[scale=0.6]{/home/hilary/OpenFOAM/hilary-dev/run/hilary/warmBubble/Boussinesq/multiFluid/singleColumn_Pi_1e4/1000/sigmaCompare} & \includegraphics[scale=0.6]{/home/hilary/OpenFOAM/hilary-dev/run/hilary/warmBubble/Boussinesq/multiFluid/singleColumn_Pi_1e4/1000/bCompare} & \includegraphics[scale=0.6]{/home/hilary/OpenFOAM/hilary-dev/run/hilary/warmBubble/Boussinesq/multiFluid/singleColumn_Pi_1e4/1000/wCompare} & \includegraphics[scale=0.6]{/home/hilary/OpenFOAM/hilary-dev/run/hilary/warmBubble/Boussinesq/multiFluid/singleColumn_Pi_1e4/1000/Pcompare}\tabularnewline
\multicolumn{4}{c}{Results at $t=1000\ \text{s}$ using $S_{ij}=-\nabla\cdot\mathbf{u}_{i}$,
$p_{i}=0$, $C_{D}/r_{c}=0$, $b_{ij}^{T}=b_{i}$}\tabularnewline
\includegraphics[scale=0.6]{/home/hilary/OpenFOAM/hilary-dev/run/hilary/warmBubble/Boussinesq/multiFluid/singleColumn_divTransfer/1000/sigmaCompare} & \includegraphics[scale=0.6]{/home/hilary/OpenFOAM/hilary-dev/run/hilary/warmBubble/Boussinesq/multiFluid/singleColumn_divTransfer/1000/bCompare} & \includegraphics[scale=0.6]{/home/hilary/OpenFOAM/hilary-dev/run/hilary/warmBubble/Boussinesq/multiFluid/singleColumn_divTransfer/1000/wCompare} & \includegraphics[scale=0.6]{/home/hilary/OpenFOAM/hilary-dev/run/hilary/warmBubble/Boussinesq/multiFluid/singleColumn_divTransfer/1000/Pcompare}\tabularnewline
\multicolumn{4}{c}{Results at $t=1000\ \text{s}$ using $S_{01}=w_{1}0.2/r_{c}$, $p_{i}=-10^{4}\nabla\cdot\mathbf{u}_{i}$,
$C_{D}=0$, $r_{c}=2000\text{m}$, $b_{ij}^{T}=b_{i}$}\tabularnewline
\includegraphics[scale=0.6]{/home/hilary/OpenFOAM/hilary-dev/run/hilary/warmBubble/Boussinesq/multiFluid/singleColumn_Pi_1e4_ent/1000/sigmaCompare} & \includegraphics[scale=0.6]{/home/hilary/OpenFOAM/hilary-dev/run/hilary/warmBubble/Boussinesq/multiFluid/singleColumn_Pi_1e4_ent/1000/bCompare} & \includegraphics[scale=0.6]{/home/hilary/OpenFOAM/hilary-dev/run/hilary/warmBubble/Boussinesq/multiFluid/singleColumn_Pi_1e4_ent/1000/wCompare} & \includegraphics[scale=0.6]{/home/hilary/OpenFOAM/hilary-dev/run/hilary/warmBubble/Boussinesq/multiFluid/singleColumn_Pi_1e4_ent/1000/Pcompare}\tabularnewline
\end{tabular}

\caption{Solid lines are the multi-fluid, single column solutions of the rising
bubble. Dashed lines are the single fluid, resolved solutions. \label{fig:multiFluidSingleColumn-bad}}
\end{figure}

\noindent 
\begin{figure}
\begin{tabular}{cccc}
$\sigma_{1}$ & $b$, $b_{0}$ and $b_{1}$ & $w_{1}$ & $P_{0}$ and $P_{1}$\tabularnewline
\multicolumn{4}{c}{Results at $t=1000\ $s using $S_{ij}=-\nabla\cdot\mathbf{u}_{i}$,
$p_{i}=-4\times10^{3}\nabla\cdot\mathbf{u}_{i}$, $C_{D}/r_{c}=0$,
$b_{ij}^{T}=b_{i}$ }\tabularnewline
\includegraphics[scale=0.6]{/home/hilary/OpenFOAM/hilary-dev/run/hilary/warmBubble/Boussinesq/multiFluid/singleColumn_Pi_4e3_divTransfer/1000/sigmaCompare} & \includegraphics[scale=0.6]{/home/hilary/OpenFOAM/hilary-dev/run/hilary/warmBubble/Boussinesq/multiFluid/singleColumn_Pi_4e3_divTransfer/1000/bCompare} & \includegraphics[scale=0.6]{/home/hilary/OpenFOAM/hilary-dev/run/hilary/warmBubble/Boussinesq/multiFluid/singleColumn_Pi_4e3_divTransfer/1000/wCompare} & \includegraphics[scale=0.6]{/home/hilary/OpenFOAM/hilary-dev/run/hilary/warmBubble/Boussinesq/multiFluid/singleColumn_Pi_4e3_divTransfer/1000/Pcompare}\tabularnewline
\multicolumn{4}{c}{Results at $t=1000\ \text{s}$ using $S_{ij}=-\nabla\cdot\mathbf{u}_{i}$,
$p_{i}=-10^{4}\nabla\cdot\mathbf{u}_{i}$, $C_{D}/r_{c}=0$, $b_{ij}^{T}=b_{i}$}\tabularnewline
\includegraphics[scale=0.6]{/home/hilary/OpenFOAM/hilary-dev/run/hilary/warmBubble/Boussinesq/multiFluid/singleColumn_Pi_1e4_divTransfer/1000/sigmaCompare} & \includegraphics[scale=0.6]{/home/hilary/OpenFOAM/hilary-dev/run/hilary/warmBubble/Boussinesq/multiFluid/singleColumn_Pi_1e4_divTransfer/1000/bCompare} & \includegraphics[scale=0.6]{/home/hilary/OpenFOAM/hilary-dev/run/hilary/warmBubble/Boussinesq/multiFluid/singleColumn_Pi_1e4_divTransfer/1000/wCompare} & \includegraphics[scale=0.6]{/home/hilary/OpenFOAM/hilary-dev/run/hilary/warmBubble/Boussinesq/multiFluid/singleColumn_Pi_1e4_divTransfer/1000/Pcompare}\tabularnewline
\multicolumn{4}{c}{Results at $t=1000\ \text{s}$ using $S_{ij}=-\nabla\cdot\mathbf{u}_{i}$,
$p_{i}=-10^{4}\nabla\cdot\mathbf{u}_{i}$, $C_{D}/r_{c}=1/2000\text{m}^{-1}$,
$b_{ij}^{T}=b_{i}$}\tabularnewline
\includegraphics[scale=0.6]{/home/hilary/OpenFOAM/hilary-dev/run/hilary/warmBubble/Boussinesq/multiFluid/singleColumn_Pi_1e4_divTransfer_drag/1000/sigmaCompare} & \includegraphics[scale=0.6]{/home/hilary/OpenFOAM/hilary-dev/run/hilary/warmBubble/Boussinesq/multiFluid/singleColumn_Pi_1e4_divTransfer_drag/1000/bCompare} & \includegraphics[scale=0.6]{/home/hilary/OpenFOAM/hilary-dev/run/hilary/warmBubble/Boussinesq/multiFluid/singleColumn_Pi_1e4_divTransfer_drag/1000/wCompare} & \includegraphics[scale=0.6]{/home/hilary/OpenFOAM/hilary-dev/run/hilary/warmBubble/Boussinesq/multiFluid/singleColumn_Pi_1e4_divTransfer_drag/1000/Pcompare}\tabularnewline
\multicolumn{4}{c}{Results at $t=1000\ \text{s}$ using $S_{ij}=-\nabla\cdot\mathbf{u}_{i}$,
$p_{i}=-10^{4}\nabla\cdot\mathbf{u}_{i}$, $C_{D}/r_{c}=0$, $b_{ij}^{T}=0$}\tabularnewline
\includegraphics[scale=0.6]{/home/hilary/OpenFOAM/hilary-dev/run/hilary/warmBubble/Boussinesq/multiFluid/singleColumn_Pi_1e4_divTransfer_bT0/1000/sigmaCompare} & \includegraphics[scale=0.6]{/home/hilary/OpenFOAM/hilary-dev/run/hilary/warmBubble/Boussinesq/multiFluid/singleColumn_Pi_1e4_divTransfer_bT0/1000/bCompare} & \includegraphics[scale=0.6]{/home/hilary/OpenFOAM/hilary-dev/run/hilary/warmBubble/Boussinesq/multiFluid/singleColumn_Pi_1e4_divTransfer_bT0/1000/wCompare} & \includegraphics[scale=0.6]{/home/hilary/OpenFOAM/hilary-dev/run/hilary/warmBubble/Boussinesq/multiFluid/singleColumn_Pi_1e4_divTransfer_bT0/1000/Pcompare}\tabularnewline
\multicolumn{4}{c}{Results at $t=1000\ \text{s}$ using $S_{ij}=-\nabla\cdot\mathbf{u}_{i}$,
$p_{i}=-10^{4}\nabla\cdot\mathbf{u}_{i}$, $C_{D}/r_{c}=0$, $b_{ij}^{T}=\frac{1}{2}\left(b_{0}+b_{1}\right)$}\tabularnewline
\includegraphics[scale=0.6]{/home/hilary/OpenFOAM/hilary-dev/run/hilary/warmBubble/Boussinesq/multiFluid/singleColumn_Pi_1e4_divTransfer_bT05/1000/sigmaCompare} & \includegraphics[scale=0.6]{/home/hilary/OpenFOAM/hilary-dev/run/hilary/warmBubble/Boussinesq/multiFluid/singleColumn_Pi_1e4_divTransfer_bT05/1000/bCompare} & \includegraphics[scale=0.6]{/home/hilary/OpenFOAM/hilary-dev/run/hilary/warmBubble/Boussinesq/multiFluid/singleColumn_Pi_1e4_divTransfer_bT05/1000/wCompare} & \includegraphics[scale=0.6]{/home/hilary/OpenFOAM/hilary-dev/run/hilary/warmBubble/Boussinesq/multiFluid/singleColumn_Pi_1e4_divTransfer_bT05/1000/Pcompare}\tabularnewline
\end{tabular}

\caption{Solid lines are the multi-fluid, single column solutions of the rising
bubble. Dashed lines are the single fluid, resolved solutions. \label{fig:multiFluidSingleColumn}}
\end{figure}

Multi-fluid, single column simulations are initialised by horizontally
and conditionally averaging the full resolution single fluid initial
conditions. Sinking air is put into fluid zero and rising air into
fluid one. The velocity field after one time step is used for initialising
the partition as the air is stationary at $t=0$. Results are shown
in figures \ref{fig:multiFluidSingleColumn-bad} and \ref{fig:multiFluidSingleColumn}
for various assumptions about transfers between fluids and various
parameter values.

If there is no drag between fluids, no transfers, no diffusion and
the pressures of both fluids are assumed equal then the multi-fluid
equations are unstable \citep{TEB19}. The simulation with no transfers
($S_{ij}=0$) in the top row of figure \ref{fig:multiFluidSingleColumn-bad}
The buoyant fluid has risen very little from the initial conditions
and oscillations have developed in all fields. $\sigma$ oscillates
between zero and one meaning that only one fluid is present at some
locations so there is no possibility for the fluids to move past each
other. Including drag (second row of figure \ref{fig:multiFluidSingleColumn-bad})
improves stability a little but does not enable the fluids to move
past each other. 

Parameterising the pressure difference between fluids as $P_{i}=P+p_{i}=P-\gamma\nabla\cdot\mathbf{u}_{i}$
with $\gamma=10^{4}\text{m}^{2}\text{s}^{-1}$ stabilises the solution
and the resulting pressure gradients force the two fluids to move
past each other (third row of figure \ref{fig:multiFluidSingleColumn-bad}).
However there is an unrealistic discontinuity with no fluid 1 below
the buoyant fluid and very little falling air at the lowest height
of the rising fluid. With a single column and two fluids, all air
must rise in one fluid and sink in the other. Continuity does not
allow for any other solution. Pressure gradients can accelerate or
decelerate this flow but as the model is set up, with no transfers
between fluids, no fluid can change direction. It is therefore essential
to have transfers between fluids.

Including entrainment and detrainment as $S_{ij}=-\nabla\cdot\mathbf{u}_{i}$
can stabilise the solution and enable the fluids to move past each
other even without pressure differences between the fluids (fourth
row of figure \ref{fig:multiFluidSingleColumn-bad}). However too
much is detrained from fluid 1, fluid 1 looses too much buoyancy and
a discontinuity arises at the bubble front; the buoyancy force is
producing vertical motion but there is no pressure to make it smooth. 

The other entrainment option is the commonly used fractional entrainment
rate, $\varepsilon=0.2/r_{c}$. This is used combined with pressure
differences between fluids in the bottom row of figure \ref{fig:multiFluidSingleColumn-bad}.
This combination produces smooth solutions but the updraught is too
weak and too much is entrained into fluid 1. Using this entrainment,
the updraught will entrain more and more air until $\sigma_{1}=1$.

Once we use entrainment and detrainment set by divergence ($S_{ij}=-\nabla\cdot\mathbf{u}_{i}$)
and pressure difference between the fluids controlled by divergence
($P_{i}=P-\gamma\nabla\cdot\mathbf{u}_{i}$) simulations are more
realistic (first two rows of figure \ref{fig:multiFluidSingleColumn}).
The detrainment of buoyant fluid leads to smooth profiles behind the
bubble although $\sigma_{1}$ is lower than the resolved simulation
in the lower half of the domain and the fluid left behind is too warm.
This is expected since the detrained fluid is assumed to have the
mean fluid properties of the fluid it is leaving ($b_{ij}^{T}=b_{i}$)
whereas in reality the least buoyant fluid will leave the rising fluid
and start to fall. Smaller values of $\gamma$ leave a discontinuity
in the updraught velocity at the leading edge of the bubble and larger
values of $\gamma$ reduce the velocity further, meaning that the
bubble does not rise enough (not shown). None of the simulations reproduce
the peak updraught velocities but the simulation with $p_{i}=-10^{4}\nabla\cdot\mathbf{u}_{i}$
and $S_{ij}=-\nabla\cdot\mathbf{u}_{i}$ comes closest, meaning that
the bubble rise is approximately correct and $\sigma_{1}$ is approximately
correct where the bubble is although it is less accurate above and
below the bubble. Using $p_{i}=-10^{4}\nabla\cdot\mathbf{u}_{i}$
produces accurate pressure increases above the centre of the bubble
but does not produce the large low pressure at the bottom of the bubble,
which is located at the lobes of the highest buoyancy in the resolved
simulation (figure \ref{fig:bubble}). It would be tempting to model
this by assuming hydrostatic pressure or the Bernouilli equation,
but these are both forms of the momentum equation which is used to
calculate the velocity and so cannot also be used to model the pressure. 

The results in the third row of figure \ref{fig:multiFluidSingleColumn}
use $p_{i}=-10^{4}\nabla\cdot\mathbf{u}_{i}$ and $S_{ij}=-\nabla\cdot\mathbf{u}_{i}$
but this time with drag set by $C_{D}/r_{c}=1/2000\text{m}^{-1}$.
The drag slows the updraughts without increasing the mixing between
the fluids which makes the solutions less like the resolved simulations.
Although drag is known to be an important term in the momentum equations
for buoyant thermals and plumes \citep[eg][]{RC15}, the drag is related
to mixing with downdraught air which is accounted for in the entrainment
term of the momentum equation. So adding a separate term from drag
risks double counting and certainly slows the plume too much. 

Two simple models for estimating the buoyancy of the fluid transferred
between updraught and downdraught have been tested. The results in
the fourth row if figure \ref{fig:multiFluidSingleColumn} assume
that the fluid transferred has zero buoyancy in order to keep the
positively buoyant air in fluid one and the negatively buoyant air
in fluid 0. Thus when zero buoyancy air leaves fluid one (the updraught),
the mean buoyancy of fluid one increases. This leads to very high
buoyancy in fluid one which is not present in the resolved simulation.
The results in the bottom row of figure \ref{fig:multiFluidSingleColumn}
use $b_{ij}^{T}=\frac{1}{2}\left(b_{0}+b_{1}\right)$ following \citet{TEB19}
so that the fluid transferred is influenced by the fluid it is moving
towards as well as where it has come from. This again leads to very
large buoyancy in fluid one and also negative buoyancy in fluid zero
which should not occur for this test case. These simple models reinforce
the idea that the buoyancy of the fluid transferred should depend
on knowledge of sub-grid-scale variability of $b$ within each fluid
\textendash{} these simple assumptions can lead to unrealistic behaviour. 

This section has demonstrated the value of using a single column,
multi-fluid model to simulate a buoyant rising bubble with a single
updraght fluid and a single downdraught fluid. Transfers between the
fluids have been successfully modelled using entrainment set by divergence
and the pressure difference between the fluids has also been represented
by divergence. There are clearly discrepancies between the single
fluid resolved rising bubble and the two fluid single column rising
bubble but we do wish aim to over-fit the multi-fluid model to one,
dry, two-dimensional test case.

\subsection{Radiative-Convective Equilibrium (RCE)}

A two-dimensional, dry radiative convective equilibrium test case
is devised to mimic some properties of atmospheric convection, in
particular the difficulty resolving the flow at coarse resolution.
The domain is 160km wide, 10km tall and is resolved by 640 cells in
the horizontal and 40 in the vertical ($\Delta x=\Delta z=250$m).
The top and bottom boundaries are zero velocity and the lateral boundaries
are periodic. A heat flux of $h=10^{-3}\text{m}^{2}\text{s}^{-3}$
is imposed at the bottom boundary leading to a boundary condition
of $\partial b/\partial z=-h/\alpha\text{s}^{-2}$ where $\alpha=100\ \text{m}^{2}\text{s}^{-1}$
is the buoyancy diffusivity. The top boundary has $\partial b/\partial z=0$.
Diffusion is applied to the momentum equation with a coefficient $\nu=70.7\ \text{m}^{2}\text{s}^{-1}$
so as to give a Plank constant of $0.707$. Uniform cooling of $-1\times10^{-7}\text{m}\text{s}^{-3}$
is applied to the domain to maintain equilibrium.

The fluid is initially stationary with $b=0$. In order to quickly
and reproducibly initialise instability, $b=10^{-4}\text{m}^{2}\text{s}^{-1}$
is imposed in a square of side length 2000m at the centre in the $x$
direction and at the ground. The simulation is run using a time step
of $5\ \text{s}$. A quasi-steady, choatic state is reached after
about $3\times10^{4}\text{s}$ and the simulation is run for a further
$2\times10^{5}\text{s}$ and conditional averages are calculated over
the final $10^{5}$s.

\subsubsection{Results of the Resolved Simulation}

This test case is designed to have strong upward velocity in narrow
plumes and weak descent elsewhere to mimic atmospheric convection
but without the complication of moisture, phase changes and three
spatial dimensions. A snapshot of the buoyancy and pressure of the
radiative-convective equilibrium test case at $2\times10^{5}\text{s}$
are shown in figure \ref{fig:RCE_resolved} with vertical velocity
contoured showing intense, narrow updraughts.

Figure \ref{fig:RCE_resolved} shows buoyant air at the ground which
rises in narrow plumes, accelerating due to buoyancy and pressure
gradients then decelerating due to pressure gradients before reaching
the top. The plumes spread out and the sinking air is a mixture of
warm and cold with similar pressure gradients in the rising and falling
air, accelerating then decelerating the flow so as to maintain continuity. 

\noindent 
\begin{figure}
\includegraphics[width=1\textwidth]{/home/hilary/OpenFOAM/hilary-dev/run/hilary/RCE/Boussinesq/hiRes_wide/200000/bw}

Buoyancy ($\text{m}\text{s}^{-2}$) \includegraphics[width=0.8\textwidth]{/home/hilary/OpenFOAM/hilary-dev/run/hilary/RCE/Boussinesq/hiRes_wide/legends/bw_b}

\includegraphics[width=1\textwidth]{/home/hilary/OpenFOAM/hilary-dev/run/hilary/RCE/Boussinesq/hiRes_wide/200000/Pw}

Pressure ($\text{m}^{2}\text{s}^{-2}$) \includegraphics[width=0.8\textwidth]{/home/hilary/OpenFOAM/hilary-dev/run/hilary/RCE/Boussinesq/hiRes_wide/legends/Pw_P}

\caption{Buoyancy, vertical velocity and pressure of resolved radiative-convective
equilibrium test case. Vertical velocity contours are black every
0.5m/s with negative contours dashed and the zero contour dotted.
Domain is 10km high and 160km wide and the z-direction is stretched
a factor of 2 in the plots. \label{fig:RCE_resolved}}
\end{figure}

Horizontal averages conditioned on $w$ and time averaged every 1000s
between $10^{5}\ \text{s}$ and $2\times10^{5}\ \text{s}$ are shown
in figure \ref{fig:RCE_hMean}. The rising and falling volume fractions,
$\sigma_{1}$ and $\sigma_{0}$, remain close to $\frac{1}{2}$ throughout
the depth. The rising plumes look narrow in figure \ref{fig:RCE_resolved}
but there are wide regions of slowly rising air around the plume cores.
Due to this entrained cool air and warm air sinking after hitting
the model top, the buoyancy of the rising and falling fluids are close
with the rising plume warmer near the surface. The air cools with
height then warms near the model top due to accumulation of warm air. 

\noindent 
\begin{figure}
\includegraphics[scale=0.7]{/home/hilary/OpenFOAM/hilary-dev/run/hilary/RCE/Boussinesq/hiRes_wide/timeMean_w/sigmaMean}\includegraphics[scale=0.7]{/home/hilary/OpenFOAM/hilary-dev/run/hilary/RCE/Boussinesq/hiRes_wide/timeMean_w/bMean}\includegraphics[scale=0.7]{/home/hilary/OpenFOAM/hilary-dev/run/hilary/RCE/Boussinesq/hiRes_wide/timeMean_w/wMean}\includegraphics[scale=0.7]{/home/hilary/OpenFOAM/hilary-dev/run/hilary/RCE/Boussinesq/hiRes_wide/timeMean_w/PMean}

\caption{Horizontal averages of the resolved RCE case averaged between $t=10^{5}\text{s}$
and $2\times10^{5}\text{s}$.  \label{fig:RCE_hMean}}
\end{figure}

The similarity of the mean values of the rising and falling fluid
is striking in the vertical velocity in figure \ref{fig:RCE_hMean}
which shows acceleration then deceleration controlled by pressure
gradients. The slightly stronger pressure gradients in the rising
fluid match the slightly higher speed. 

\subsubsection{Results of the Single Column Simulations}

The tow fluid model is used to simulate the RCE test case in a single
column of 40 cells with zero buoyancy gradient at the top and a buoyancy
gradien5t of $10^{-5}\text{s}^{-2}$ at the ground for both fluids.
The two fluids are initilasied usign the time mean conditional horizontal
averages from the single fluid resolved simulation. The entrainment
model $S_{ij}=-\nabla\cdot\mathbf{u}_{i}$ is used for all simulations
to account for transfers throughout the domain to represent cloud
base mass flux (at the bottom boundary), cloud top detrainment and
lateral entrainment. 

Results of simulations using two values of $\gamma$, two models for
$b_{ij}^{T}$ and with and without drag are shown in figure \ref{fig:RCE_singleColumn}.
Solid lines show results of the single column, multifluid model at
$t=10^{4}\text{s}$ and dashed lines show the time averaged, horizontally
conditionally averaged resolved single fluid simulation. The entrainment
model $S_{ij}=-\nabla\cdot\mathbf{u}_{i}$ keeps $\sigma_{i}$ at
about 0.5 throughout the depth, as in the resolved simulatin, with
entrainemnt in the lower half of the domain ($S_{01}>0$) and detrainment
in the upper half, where the air decelerates towards the top ($S_{10}>0$).
All of the two fluid simulations maintain a larger buoyancy difference
between the two fluids than the resolved simulation; the single column
solution cannot capture the complex distribution of heat in both fluids.
The single column solutions accurately capture the large buoyancy
gradients near the ground and near uniform profile above this. The
model $b_{ij}^{T}=\frac{1}{2}(b_{0}+b_{1})$ represents the buoyancy
of the transferred fluikd very badly, leading to strong cooling in
fluid zero near the ground due to the large transferts. It is clearly
more accurate to assume that fluids take their mean propertieds when
transferred, $b_{ij}^{T}=b_{i}$. The vertical velocity is sensitive
to the pressure differences between fluids with $\gamma=2000\text{m}^{2}\text{s}^{-1}$
giving the closest match. 

\noindent 
\begin{figure}
\begin{tabular}{ccccc}
\multicolumn{5}{c}{Single column with $S_{ij}=-\nabla\cdot\mathbf{u}_{i}$ and $b_{ij}^{T}=b_{i}$,
$\gamma=4\times10^{3}\text{m}^{2}\text{s}^{-1}$ versus resolved}\tabularnewline
$\sigma$ & $\sigma S\ (\text{s}^{-1})$  & $b\ (\text{m}\ \text{s}^{-2})$ & $w\ (\text{m}\ \text{s}^{-1})$ & $P\ (\text{m}^{2}\text{s}^{-2})$\tabularnewline
\includegraphics[scale=0.7]{/home/hilary/OpenFOAM/hilary-dev/run/hilary/RCE/Boussinesq/singleColumn/divTransfer_Pi4e3_w/100000/sigma} & \includegraphics[scale=0.7]{/home/hilary/OpenFOAM/hilary-dev/run/hilary/RCE/Boussinesq/singleColumn/divTransfer_Pi4e3_w/100000/massTransfer} & \includegraphics[scale=0.7]{/home/hilary/OpenFOAM/hilary-dev/run/hilary/RCE/Boussinesq/singleColumn/divTransfer_Pi4e3_w/100000/b} & \includegraphics[scale=0.7]{/home/hilary/OpenFOAM/hilary-dev/run/hilary/RCE/Boussinesq/singleColumn/divTransfer_Pi4e3_w/100000/u} & \includegraphics[scale=0.7]{/home/hilary/OpenFOAM/hilary-dev/run/hilary/RCE/Boussinesq/singleColumn/divTransfer_Pi4e3_w/100000/P}\tabularnewline
\multicolumn{5}{c}{Single column with $S_{ij}=-\nabla\cdot\mathbf{u}_{i}$ and $b_{ij}^{T}=b_{i}$,
$\gamma=2\times10^{3}\text{m}^{2}\text{s}^{-1}$ versus resolved}\tabularnewline
\includegraphics[scale=0.7]{/home/hilary/OpenFOAM/hilary-dev/run/hilary/RCE/Boussinesq/singleColumn/divTransfer_Pi2e3_w/100000/sigma} & \includegraphics[scale=0.7]{/home/hilary/OpenFOAM/hilary-dev/run/hilary/RCE/Boussinesq/singleColumn/divTransfer_Pi2e3_w/100000/massTransfer} & \includegraphics[scale=0.7]{/home/hilary/OpenFOAM/hilary-dev/run/hilary/RCE/Boussinesq/singleColumn/divTransfer_Pi2e3_w/100000/b} & \includegraphics[scale=0.7]{/home/hilary/OpenFOAM/hilary-dev/run/hilary/RCE/Boussinesq/singleColumn/divTransfer_Pi2e3_w/100000/u} & \includegraphics[scale=0.7]{/home/hilary/OpenFOAM/hilary-dev/run/hilary/RCE/Boussinesq/singleColumn/divTransfer_Pi2e3_w/100000/P}\tabularnewline
\multicolumn{5}{c}{Single column with $S_{ij}=-\nabla\cdot\mathbf{u}_{i}$ and $b_{ij}^{T}=\frac{1}{2}\left(b_{i}+b_{j}\right)$,
$\gamma=2\times10^{3}\text{m}^{2}\text{s}^{-1}$ versus resolved}\tabularnewline
\includegraphics[scale=0.7]{/home/hilary/OpenFOAM/hilary-dev/run/hilary/RCE/Boussinesq/singleColumn/divTransfer_Pi2e3_bT05_w/100000/sigma} & \includegraphics[scale=0.7]{/home/hilary/OpenFOAM/hilary-dev/run/hilary/RCE/Boussinesq/singleColumn/divTransfer_Pi2e3_bT05_w/100000/massTransfer} & \includegraphics[scale=0.7]{/home/hilary/OpenFOAM/hilary-dev/run/hilary/RCE/Boussinesq/singleColumn/divTransfer_Pi2e3_bT05_w/100000/b} & \includegraphics[scale=0.7]{/home/hilary/OpenFOAM/hilary-dev/run/hilary/RCE/Boussinesq/singleColumn/divTransfer_Pi2e3_bT05_w/100000/u} & \includegraphics[scale=0.7]{/home/hilary/OpenFOAM/hilary-dev/run/hilary/RCE/Boussinesq/singleColumn/divTransfer_Pi2e3_bT05_w/100000/P}\tabularnewline
\end{tabular}

\caption{Results of single Column RCE simulations after $10^{4}\ \text{s}$.
Dashed lines show time averaged, conditionally averaged horizontal
means of the resolved simulations. \label{fig:RCE_singleColumn}}
\end{figure}


\section{Summary, Conclusions and Further Work}

\section*{Acknowledgements}

Many thanks to Peter Clark, John Thuburn and Chris Holloway for valuable
discussions and proof reading. Thanks to the NERC/Met Office Paracon
project. We acknowledge funding from the Circle-A and RevCon Paracon
projects NE/N013743/1 and NE/xxx.

\bibliographystyle{abbrvnat}
\bibliography{numerics}

\end{document}
