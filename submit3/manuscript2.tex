%%%%%%%%%%%%%%%%%%%%%%%%%%%%%%%%%%%%%%%%%%%%%%%%%%%%%%%%%%%%%%%%%%%%%%%%%%%%
% AGUJournalTemplate.tex: this template file is for articles formatted with LaTeX
%
% This file includes commands and instructions
% given in the order necessary to produce a final output that will
% satisfy AGU requirements, including customized APA reference formatting.
%
% You may copy this file and give it your
% article name, and enter your text.
%
%
% Step 1: Set the \documentclass
%
%

%% To submit your paper:
\documentclass[draft]{agujournal2019}
\usepackage{url} %this package should fix any errors with URLs in refs.
\usepackage{lineno}
%\usepackage[inline]{trackchanges} %for better track changes. finalnew option
\usepackage[finalnew]{trackchanges}
\soulregister\ref7
\soulregister\cite7
\linenumbers
%%%%%%%
% As of 2018 we recommend use of the TrackChanges package to mark revisions.
% The trackchanges package adds five new LaTeX commands:
%
%  \note[editor]{The note}
%  \annote[editor]{Text to annotate}{The note}
%  \add[editor]{Text to add}
%  \remove[editor]{Text to remove}
%  \change[editor]{Text to remove}{Text to add}
%
% complete documentation is here: http://trackchanges.sourceforge.net/
%%%%%%%
\addeditor{HW}
\addeditor{WM}

\draftfalse

\journalname{Journal of Advances in Modeling Earth Systems (JAMES)}

\usepackage{mathptmx}
\usepackage{array}
\usepackage{rotating}
\usepackage{amsmath}
\usepackage{amssymb}
\usepackage{graphicx}

\begin{document}

\title{Multi-fluids for Representing Sub-filter-scale Convection}

\authors
{
    Hilary Weller\affil{1}\thanks{Meteorology, University of Reading, UK}
    ,
    William McIntyre\affil{1}
    ,
    Daniel Shipley\affil{1}
}

\affiliation{1}{Meteorology, University of Reading, UK}

\correspondingauthor{Hilary Weller}{h.weller@reading.ac.uk}

\begin{keypoints}
\item Multi-fluid modelling enables net mass transport by \change[HW]{parameterised}{sub-grid-scale} convection and non-equlibrium.
\item Multi-fluid modelling means a new dynamical core rather than a parameterisation for an existing core
\item Closures for the pressure difference between fluids and entrainment and detrainment are provided
\end{keypoints}

\begin{abstract}
Multi-fluid modelling is a promising new method of representing sub- and near-grid-scale convection allowing for consistent and accurate
treatment of net mass transport by convection \add[HW]{and of} non-equilibrium \add[HW]{dynamics} \remove[HW]{and turbulence}. The air is partitioned into \change[HW]{convecting and non-convecting}{two or more}
fluids \add[HW]{which may represent, for example, updrafts and the stable environment.} \change[HW]{each with their}{Each fluid has its} own velocity, temperature and constituents
with separate equations of motion. \remove[HW]{can be solved for each fluid,
thus representing features that cannot be resolved in a single-fluid
at coarse resolution.} 

This paper presents \change[HW]{multi}{two}-fluid Boussinesq equations for representing under-resolved dry convection
\add[HW]{with sinking and $w=0$ air in fluid 0 and rising air in fluid 1}. \change[HW]{We present a new}{A} model for entrainment
and detrainment based on divergence \remove[HW]{which} leads to excellent representation of the convective area fraction.
Previous multi-fluid modelling of convection has
\change[HW]{assumed that all fluids share the same pressure.}{used the same pressure for both fluids.}
\change[HW]{We show that this is}{This is shown to be} a bad \change[HW]{assumption}{approximation} and \remove[HW]{we propose and validate} a model
for the pressure difference between the fluids based on divergence \add[HW]{is presented}. 

Two vertical slice test cases are \change[HW]{used}{developed} to \add[HW]{tune parameters and to} \change[HW]{validate}{evaluate} the \change[HW]{multi}{two}-fluid \change[HW]{model}{equations for representing dry, sub-grid-scale convection}: a buoyant rising bubble and radiative convective equilibrium. \add[HW]{These}
are \add[HW]{first} simulated at high resolution and conditionally averaged based
on the sign of the vertical velocity. \add[HW]{The test cases are next simulated with the two-fluid model which} \remove[HW]{The multi-fluid model} reproduces
the mean properties of the rising and falling fluids in \change[HW]{a single}{one} column. 
\end{abstract}

\section*{Plain Language Summary}
Clouds and buoyant convection are often smaller than the grid size in weather and climate prediction models but they are of central importance. Therefore their effects are parameterised -- their interactions with the larger scales are estimated based on properties of the larger scales. Convection parameterisations are a large source of error in models of the atmosphere despite decades of effort to improve them. 
\change[HW]{An important assumption in most convection parameterisations is that convection moves heat and moisture upwards but not mass. This makes }{Some unrealistic assumptions remain such as that convection does not transport mass upwards, convection is in equilibrium with the surroundings and that the variations in the horizontal direction are not relevant. These make}
it much simpler to incorporate convection parameterisations into existing models. 
Multi-fluid modelling
\change[HW]{will enable mass transport by convection}{does not rely on these assumptions}
but the approach means a whole new atmospheric model rather than a stand alone routine that can be incorporated into an existing model. This paper provides some closures necessary for multi-fluid modelling and \change[HW]{validation}{evaluation} of the technique for two dimensional dry convection.

\section{Introduction \label{sec:intro}}

Slow progress has been made improving the parameterisation of sub-grid-scale
convection since \citeA{AS74}, \citeA{Tied89}, \citeA{GR90} and \citeA{KF90}. Convection is still
one of the weakest aspects of large scale models of the atmosphere
\cite{HPB+14,SAB+13,ipcc41}. Improvements have been made \add[HW]{by} relaxing
the quasi-equilibrium assumption \cite{PR98,GG05,Par14}, including
stochasticity \cite{PC08} and allowing net mass transport by convection
\cite{KB08,MB19}. These developments have enabled limited simulation
in the grey zone, where convection is partially resolved. However
there are still systematic biases implying that improvements should
still be possible.

\add[HW]{\protect\citeA{LR01} introduced the ``assumed-distribution higher-order closure'' (ADHOC) parameterisation of the boundary layer and convection which solves prognostic equations for higher-order moments of the sub-grid-scale flow, such as the triple correlation of the vertical velocity anomalies, and diagnoses updraft and downdraft means from the higher-order moments. Thus all necessary statistics can be calculated from the grid-box-mean values that are predicted by the dynamical core. Net mass transport by convection is included because it is part of the grid-box-mean. ADHOC was extended by \protect\citeA{GLC02}
to use bi-Gaussian probability distribution functions rather than top-hat distributions to represent sub-grid-scale variability. This formed the basis of the ``Cloud Layers Unified By Binormals'' (CLUBB) parameterization 
\protect\cite<>{LSW+12}
 which was first used at grey zone resolutions.}

\citeA{TWV+18} proposed multi-fluid modelling of convection in order
to treat non-equilibrium and net mass transport \remove[HW]{correctly and} consistently.
\add[HW]{The multi-fluids approach has a potential advantage over ADHOC and CLUBB in solving prognostic equations for first-order moments directly (mean properties of e.g. updraft and environment) rather than solving prognostic equations for higher-order moments of the whole fluid in order to diagnose means of updrafts and the environment. Solving directly for first-order moments in each fluid could be less sensitive to closure assumptions.}
\add[HW]{The multi-fluids approach has an advantage over prognostic plume models such as \protect\citeA{GG05} as multi-fluids enables the mass transported by convection to feed back onto the continuity equation.}
\remove{with the same numerics and physics to prognose properties of separate parts of the fluid, for example updrafts, the stable environment and downdrafts dynamics}. \citeA{Yano14} derived similar
equations to \citeA{TWV+18} \remove[HW]{for multiple segments} and showed how this is a generalisation
of the mass-flux formulation. The \citeA{TKP+18} ``extended eddy
diffusivity mass flux'' or extended EDMF scheme \remove[HW]{also} uses similar
equations but \citeA{TKP+18}
\change[HW]{didn't propose a suitable numerical method for solving the equations.}{only solve the equations in a single column without coupling to the continuity equation of a dynamical core.}

\change[HW]{Multi-fluids promises big improvements in the representation of sub-grid-scale convection and unification with the representation of turbulence, particularly at the grey zone. However, there is a long way to go, particularly as multi-fluids}{The disadvantage of the multi-fluids approach is that it} requires
a dedicated model rather than being a stand alone parameterisation
for a dynamical core. \citeA{TEB19} and \citeA{WM19} presented solutions
of \change[HW]{the multi}{two}-fluid equations
\change[HW]{assuming two fluids that share}{using} the same pressure
\add[HW]{for both fluids}.
\remove[HW]{(or that pressure differences are predominantly due to drag
between the fluids)}. \citeA{TEB19} showed that these equations \remove[HW]{(without
drag)} are in fact unstable and can be stabilised by diffusion of vertical
velocity whereas \citeA{WM19} stabilised the equations with diffusion
and drag between fluids. Neither of these stabilisation techniques
are acceptable;
\add[HW]{sufficient diffusion to stabilise the multi-fluid equations may smooth out real features of the flow and the diffusion between fluids and drag used by \protect\citeA{WM19} meant that both fluids tended to move together rather than through each other.}
In this paper, \remove[HW]{we show that there are} significant
and sustained pressure differences between the fluids \add[HW]{are shown to exist} (section \ref{sec:results}) \add[HW]{and that the use of separate pressures stabilises the equations}. 

This paper presents the multi-fluid Boussinesq equations for representing
sub-grid-scale convection including a parameterisation of the pressure
difference between fluids (section \ref{subsec:fluidPressure}) and
a new model of entrainment and detrainment (section \ref{subsec:Sij}).
The numerical method for solving the equations is described in section
\ref{sec:numerics}. \change[HW]{We compare results of resolved dry convection
in a single-fluid two dimensional Boussinesq model with a two-fluid,
single column model in section {\protect\ref{sec:results}} using two vertical slice test cases. }{Results of two vertical slice test cases are presented in section {\protect\ref{sec:results}}. These are first simulated at high resolution in a single fluid model and then statistics of the flow are reproduced solving the two-fluid equations in one column.}

\section{The Multi-fluid Boussinesq Equations \label{sec:multiBouEqns}}

\subsection{\add[HW]{Derivation}}

Many of the challenges of representing convection with the multi-fluid
Navier-Stokes equations \cite<e.g. those described in>{WM19} carry
over to the Boussinesq equations \add[HW]{because} convection is buoyancy dominated
and close to incompressible. The resolved test cases presented in
section \ref{sec:results} give very similar results for Navier-Stokes
and Boussinesq equations. Therefore\change[WM]{ for simplicity we focus on the}{, the}
\remove[HW]{multi-fluid} Boussinesq equations without background stratification \add[WM]{are chosen for simplicity}:
\begin{eqnarray}
\frac{\partial \mathbf{u}}{\partial t} + \nabla\cdot(\mathbf{u}\mathbf{u})
+\nabla P & = & b \mathbf{k}+\nu\nabla^{2}\mathbf{u}
\label{eq:singleMom}
\\
\frac{\partial b}{\partial t} + \nabla\cdot(\mathbf{u}b)
 & = & \alpha\nabla^{2}b
\label{eq:singleb}
\\
\nabla\cdot\mathbf{u} & = & 0
\label{eq:singleDivFree}
\end{eqnarray}
where variables are defined in table \ref{tab:defns}. \add[HW]{Note that the ``pressure'' variable, $P$, is actually pressure divided by a constant reference density.}

\begin{table}
\begin{tabular}{c>{\raggedright}p{0.85\textwidth}}
$\sigma_{i}$ & Volume fraction of fluid $i$ so that $\sum_{i}\sigma_{i}=1$
\tabularnewline
\add[HW]{$\mathbf{u}$} & 
\add[HW]{Velocity of the single-fluid equations  ($\text{m}\text{s}^{-1}$) and velocity averaged over all fluids of the multi-fluid equations}
\tabularnewline
$\mathbf{u}_{i}$ & Velocity of fluid $i$ ($\text{m}\text{s}^{-1}$)
\tabularnewline
$w_{i}$ & Vertical component of $\mathbf{u}_{i}$ ($\text{m}\text{s}^{-1}$)\tabularnewline
$S_{ij}$ & Rate of volume fraction transfer from fluid $i$ to $j$ ($\text{s}^{-1}$)
\tabularnewline
\add[HW]{$P$} & \add[HW]{Perturbation pressure of the single-fluid equations $=p^{\prime}/\rho_{r}$ ($\text{m}^{2}\text{s}^{-2}$) and pressure averaged over all fluids of the multi-fluid equations}
\tabularnewline
$P_i$ & Perturbation pressure of fluid $i$
\tabularnewline
\add[HW]{$p_i$} & \add[HW]{$P_i - P$}
\tabularnewline
$\psi_{ij}^{T}$ & Value of variable $\psi$ transferred from fluid $i$ to $j$
\tabularnewline
$b$ & Buoyancy of single-fluid equations defined as -$g\rho^{\prime}/\text{\ensuremath{\rho}}_{r}$
where $\rho^{\prime}$ are departures in density from a horizontally
uniform reference, $\rho_{r}$ ($\text{m}\text{s}^{-2}$) \add[HW]{and buoyancy averaged over all fluids of the multi-fluid equations}
\tabularnewline
$b_{i}$ & Buoyancy of fluid $i$ ($\text{m}\text{s}^{-2}$)
\tabularnewline
$D_{i}\big/Dt$ & Total derivative with respect to fluid $i$ $=\partial/\partial t+\mathbf{u}_{i}\cdot\nabla$
\tabularnewline
$\mathbf{D}_{ij}$ & Drag on fluid $i$ from fluid $j$ ($\text{m}\text{s}^{-2}$)
\tabularnewline
$\gamma$ & \change[HW]{Coefficient for setting the pressure local for each fluid}{Compressibility for calculating $p_i$} ($\text{m}^{2}\text{s}^{-1}$)
\tabularnewline
$C_{D}$ & Drag coefficient $=0$ or 1
\tabularnewline
$r_{c}$ & Plume radius used for defining the drag between fluids (m)\tabularnewline
$\alpha$ & Diffusivity of buoyancy ($\text{m}^{2}\text{s}^{-1}$)\tabularnewline
$\nu$ & Viscosity ($\text{m}^{2}\text{s}^{-1}$)
\tabularnewline
\add[WM]{$\mathbf{k}$} & \add[WM]{Unit vector in the $z$-direction.}
\tabularnewline
\end{tabular}
\caption{Definitions of variables for the single- and multi-fluid Boussinesq equations.\label{tab:defns}}
\end{table}

\note[HW]{Additional material until the start of 2.2}

The single-fluid Boussinesq equations will be conditionally averaged following \citeA{TWV+18} to give the multi-fluid Boussinesq equations. Lagrangian labels, $I_i$, pick out different fluid types or regions. $I_i=1$ in fluid $i$ and 0 elsewhere. $I_i$ satisfies Lagrangian conservation without transfers:
\begin{equation}
\frac{\partial I_i}{\partial t} + \mathbf{u}\cdot\nabla I_i = 0.
\label{eq:LagrangianLabel}
\end{equation}
It is more straightforward to introduce transfers after volume averaging because the transfer terms for (\ref{eq:LagrangianLabel}) involve delta functions (this will be addressed in future publications). 
By applying a volume average (denoted by the overbar), the fluid $i$ volume fractions, velocities and buoyancies are defined by:
\begin{eqnarray}
\sigma_i &=& \overline{I_i} \\
\sigma_i \mathbf{u}_i &=& \overline{I_i \mathbf{u}} \\
\sigma_i b_i &=& \overline{I_i b}
\label{eq:defineFluidFields}
\end{eqnarray}
which implies
\begin{equation}
\sum_{i}\sigma_{i}  =  1.
\label{eq:sumOne}
\end{equation}
As the system is governed by the Boussinesq equations, it is assumed that variations in density only affect the definition of the buoyancy terms (table \protect\ref{tab:defns}) so $\mathbf{u}_i$ and $b_i$ are defined without density weighting. 

To derive a transport equation for $\sigma_i$, (\ref{eq:LagrangianLabel}) is combined with $I_i$ times (\ref{eq:singleDivFree}) and volume averaged. 
It is necessary to assume that
operators permute with the averaging \cite<as described by>{TWV+18}. Transfers between fluids are included at this stage:
\begin{equation}
\frac{\partial\sigma_{i}}{\partial t}+\nabla\cdot\sigma_{i}\mathbf{u}_{i}  =  \sum_{j\ne i}\left\{ \sigma_{j}S_{ji}-\sigma_{i}S_{ij}\right\}
\label{eq:sigma}
\end{equation}
where $\sigma_{i}S_{ij}$ is the transfer rate from fluid $i$ to $j$.

Next the conditional averaging of terms in the momentum (\ref{eq:singleMom}) and buoyancy (\ref{eq:singleb}) equations is considered. Conditionally averaging the advection terms leads to sub-filter-scale fluxes, $F_{SF}$, since the velocity and buoyancy are not uniform within each fluid at the filter scale:
\begin{eqnarray}
\overline{I_i \nabla \cdot (\mathbf{u} \mathbf{u})} &=& \nabla \cdot (\sigma_i \mathbf{u}_i  \mathbf{u}_i) + F_\text{SF}^{\mathbf{u}_i} \\
\overline{I_i \nabla (\mathbf{u} b)} &=&  \nabla \cdot (\sigma_i \mathbf{u}_i b_i) + F_\text{SF}^{b_i}
\label{eq:filterAdvection}
\end{eqnarray}
The sub-filter-scale fluxes are represented by constant viscosity diffusion terms for each fluid. The diffusion terms are written so that, in the absence of transfers between fluids and if both fluids are initialised to be identical, the fluids remain identical and $\sigma_i$ acts as a passive tracer:
\begin{eqnarray}
\overline{I_i \nu \nabla^2 \mathbf{u}} &\approx& \sigma_i \nu \nabla^2 \mathbf{u}_i \\
\overline{I_i \alpha \nabla^2 b} &\approx& \sigma_i \alpha \nabla^2b_i .
\end{eqnarray} 
%
The pressures within each fluid are defined as:
\begin{equation}
\sigma_i P_i = \overline{I_i P}.
\end{equation}
The pressure gradient can then be decomposed as:
\begin{equation}
\overline{I_i \nabla P} = \sigma_i \nabla P_i + 
    \left[ P_i\nabla\sigma_i - \overline{P\nabla I_i} \right]
\end{equation}
The terms in square brackets are the difference between the pressure force pushing outward from the fluid calculated using resolved variables and the full resolved version of the same force, both of which are related to drag. They will be represented by a drag, $\mathbf{D}_{ij}$. However this drag is likely to be smaller than the drag that is used, for example, in multi-phase modelling \cite<e.g.>{GBB+07} as it is representing the difference between two representations of drag, rather than the whole drag. Sensitivity to $\mathbf{D}_{ij}$ will be investigated in section \ref{sec:results}. 

Combining all terms gives the conditionally averaged momentum and buoyancy equations:
\begin{eqnarray}
\frac{\partial\sigma_i\mathbf{u}_{i}}{\partial t} + 
\nabla\cdot(\sigma_i \mathbf{u}_{i} \mathbf{u}_{i})
+ \sigma_i \nabla P_{i}
& = &
\sigma_i b_{i}\mathbf{k}
+
\nu\sigma_i\nabla^{2}\mathbf{u}_{i}
+
\sum_{j\ne i}\left\{
     \sigma_{j} S_{ji}\mathbf{u}_{ji}^{T}
   - \sigma_{i} S_{ij}\mathbf{u}_{ij}^{T}
   - \sigma_{i}\mathbf{D}_{ij}\right\}
\label{eq:momFlux}\\
\frac{\partial\sigma_i b_{i}}{\partial t} + 
\nabla\cdot(\sigma_i \mathbf{u}_{i} b_{i})
& = &
\alpha\sigma_i\nabla^{2}b_{i}
+
\sum_{j\ne i}\left\{
    \sigma_{j} S_{ji} b_{ji}^{T}
  - \sigma_{i} S_{ij} b_{ij}^{T}
\right\}.
\label{eq:bFlux}
\end{eqnarray}
These equations include mass transfers between fluids with $\mathbf{u}_{ij}^T$ being the velocity of the fluid transferred from $i$ to $j$ and $b_{ij}^T$ being the buoyancy of the fluid transferred from $i$ to $j$. 

Eqns (\ref{eq:momFlux}) and (\ref{eq:bFlux}) are solved in advective form in which the mass transfer terms are more complicated: \remove[HW]{due to the mass transfer terms in \protect\ref{eq:sigma}}
\begin{eqnarray}
\frac{D_{i}\mathbf{u}_{i}}{Dt}+\nabla P_{i}
& = &
b_{i}\mathbf{k}+\nu\nabla^{2}\mathbf{u}_{i}+\sum_{j\ne i}\left\{ \frac{\sigma_{j}}{\sigma_{i}}S_{ji}\left(\mathbf{u}_{ji}^{T}-\mathbf{u}_{i}\right)-S_{ij}(\mathbf{u}_{ij}^{T}-\mathbf{u}_{i})-\mathbf{D}_{ij}\right\}
\label{eq:mom}\\
\frac{D_{i}b_{i}}{Dt}
& = &
\alpha\nabla^{2}b_{i}+\sum_{j\ne i}\left\{ \frac{\sigma_{j}}{\sigma_{i}}S_{ji}\left(b_{ji}^{T}-b_{i}\right)-S_{ij}\left(b_{ij}^{T}-b_{i}\right)\right\} \label{eq:b}
\end{eqnarray}
where $D_{i}\big/Dt=\partial/\partial t+\mathbf{u}_{i}\cdot\nabla$ is the total derivative with respect to fluid $i$, 

Finally the multi-fluid version of incompressibility is
\begin{equation}
\sum_{i}\nabla\cdot\sigma_{i}\mathbf{u}_{i}  = \nabla\cdot\sum_{i}\sigma_{i}\mathbf{u}_{i}=0.
\label{eq:divFree}
\end{equation}

\subsection{Pressure of each fluid \label{subsec:fluidPressure}}

\citeA{TEB19} and \citeA{WM19}
\change[HW]{assumed that each co-located fluid shared the same pressure leading.}{used the same pressure for both fluids.}
\add[HW]{This led} to unstable equations that \citeA{TEB19}
stabilised using diffusion of vertical velocity and \citeA{WM19}
stabilised using mass exchanges or drag between the fluids. \citeA{TEB19}
achieved realistic results in \change[HW]{single}{one} column experiments of \add[HW]{the} convective
boundary layer but the amount of diffusion needed would spoil simulations
in the free troposphere \add[HW]{by diffusing real features as well as spurious artifacts}. The stabilisation used by \citeA{WM19} meant
that the two fluids tended to move as one which defeats the purpose
of using multi-fluid modelling. 

The instabilities reported by \citeA{WM19} showed growing divergence
in a fluid with vanishing volume fraction. \change[HW]{It is mass convergence
rather than velocity convergence that leads to pressure anomalies
that remove the convergence. Therefore if the volume (or mass) fraction
of fluid is vanishingly low then convergence cannot lead to an anomaly
in the total pressure.}{For single-fluid equations, divergence would lead to pressure anomalies which would act as a negative feedback on the divergence. However for multi-fluids with just one pressure, if one of the volume fractions is too small then it cannot force pressure anomalies big enough to stabilise the divergence. Use of a separate pressure for each fluid is therefore proposed to ensure stability.}

\change[WM]{We will assume that}{It is assumed that the} pressure in each fluid is given by a \remove[HW]{parameterised}
perturbation, $p_{i}$, from the total pressure, $P$:
\begin{equation}
P_{i}=P+p_{i}.
\end{equation}
\add[HW]{For stability and to respect $\sum_i \nabla\cdot\sigma_i\mathbf{u}_i=0$, we propose four constraints on the pressure perturbations, $p_i$:}
\begin{enumerate}
\item $\sum_{i}\sigma_{i}p_{i}=0$.
\item $p_i=0$ when $\mathbf{u}_{i}=\mathbf{u}$ so that pressure perturbations do not force individual fluid anomalies away from the mean.
\item $p_i \not\to 0$ when $\sigma_i\to 0$ so that pressure perturbations control divergence in vanishing fluids.
\item $p_i\not\to\infty$ as $\sigma_j\to 0$ for any combination of $i,j$.
\end{enumerate}


\add[HW]{In order to derive a suitable form of parameterisation for $p_i$ we will first consider the} \citeA{BN86} \add{model of multi-phase flow.}

\subsubsection{\add[HW]{The \protect\citeA{BN86} model}}

\citeA{BN86} proposed a parameterisation for pressure differences
in a multi-phase explosion model that has been widely used in mathematics
and engineering for simulating, \add[HW]{for example}, fluidised beds of granular material \cite<e.g.>{EHM92} and two compressible fluids \cite<e.g.>{SA99}. \add[HW]{This model is not used here but is examined in order to derive a parameterisation of pressure differences between fluids suitable for atmospheric convection.}
\citeA{SA99} wrote transport equations for both the volume fraction,
$\sigma_{i}$, and for the mass fraction,
$\sigma_{i}\rho_{i}$, without transfers between fluids which in our notation are:
\begin{eqnarray}
\frac{\partial\sigma_{i}}{\partial t}+\mathbf{V}\cdot\nabla\sigma_{i} & = & \mu\left(P_{i}-P_{j}\right)
\label{eq:BNcompaction}
\\
\frac{\partial\sigma_{i}\rho_{i}}{\partial t}+\nabla\cdot\left(\sigma_{i}\rho_{i}\mathbf{u}_{i}\right) & = & 0
\label{eq:rhoTransport}
\end{eqnarray}
where $\mathbf{V}$ is the velocity of the interface between fluids,
$\mu$ is the compaction viscosity which controls how quickly the
pressure of each fluid relaxes to the mean pressure and index $j$
refers to the other fluid in a two fluid system.
\remove[HW]{In our Boussinesq system, $\rho_{i}$ does not appear in the continuity equation.}
Different authors make different assumptions about the interface velocity.
\change[HW]{but if we assume that $\mathbf{V}=\sum_{i}\sigma_{i}\mathbf{u}_{i}$ and $\rho_{i}$ is constant then eqns (\ref{eq:divFree}), (\ref{eq:sumOne}), (\ref{eq:BNcompaction}) and (\ref{eq:rhoTransport}) imply that $\mu p_{i}=-\sigma_{i}^{2}\nabla\cdot\mathbf{u}_{i}$. If we assume that $\mathbf{V}=\mathbf{u}_{i}$ for each fluid then we get $\mu p_{i}=-\sigma_{i}\nabla\cdot\mathbf{u}_{i}$.}{Assuming uniform density, the combination of equations (\ref{eq:BNcompaction}) and (\ref{eq:rhoTransport}) (to eliminate $\partial\sigma_i/\partial t$) shows that the pressure difference between fluids is related to divergence.}

\subsubsection{\add[HW]{Proposed Model for Pressure Differences between Fluids}}

\citeA{TEB19} stabilised the multi-fluid equations by adding a diffusion
term $\nu\partial w^{2}/\partial z^{2}$ to the RHS of the $w$ equation.
If a fluid perturbation pressure is set to \change[WM]{$-\nu\partial w/\partial z$}{$p_i=-\nu\partial w/\partial z$}
\add[WM]{then} this is equivalent to the diffusion term in one dimension. However adding an \add[HW]{artificial} diffusion term will harm three dimensional simulations of
resolved convection in the free troposphere whereas adding a fluid
perturbation pressure will not. Inspired by both \citeA{BN86} and
\citeA{TEB19}
\change[WM]{we assume that the pressure perturbation in each fluid is given by}{and obeying the constraints given above, the pressure in each fluid is chosen to be}:
\begin{equation}
p_{i}=
-\underbrace{\gamma\nabla\cdot\mathbf{u}_{i}}_{\text{stabilisation}}
+
\underbrace{\gamma\sum_j \sigma_j\nabla\cdot\mathbf{u}_j}_{\text{correction}}
\label{eq:Pi_div}
\end{equation}
where $\gamma$ is a \change[HW]{free coefficient}{compressibility with units $\text{m}^2 \text{s}^{-1}$. The ``stabilistation'' term reduces to that of \protect{\citeA{TEB19}} in one dimension and the ``correction'' term ensures that $\sum_i \sigma_i p_i =0$. It should be noted that this parameterisation destroys energy conservation.}

\change[HW]{Dimensional}{Scale} analysis in section
\ref{subsec:dimAnal} will inform the choice of $\gamma$ and sensitivity
to $\gamma$ will be evaluated in section \ref{sec:results}.

\subsection{Entrainment and Detrainment \label{subsec:Sij}}

\add[HW]{It is useful to compare the transfer terms, $S_{ij}$, with models of entrainment, $\varepsilon$, and detrainment, $\delta$ used in convection parameterisations. If fluid 0 is the environment and fluid 1 is updrafts then the relationship is:}
\begin{equation}
S_{01} = w_1\varepsilon, \;\;\;\;\;\;\;
S_{10} = w_0\delta.
\end{equation}
\add[HW]{Results using two formulations of entrainment are presented in section {\protect\ref{sec:results}}. The first is similar to}
\remove[HW]{We consider dry air without turbulence so use}
the dynamic entrainment described by
\citeA{HC51}, \citeA{AK67} and \citeA{DBF+13}:
\begin{equation}
S_{ij}=\begin{cases}
-\nabla\cdot\mathbf{u}_{i} & \text{if }\nabla\cdot\mathbf{u}_{i}<0\\
0 & \text{otherwise.}
\end{cases}
\label{eq:Sdiv}
\end{equation}
\change[HW]{which is necessary to stop air in}{This formulation prevents} one fluid from accumulating when \change[WM]{vertical motion ceases}{a fluid decelerates} (at the top of a rising plume or at the bottom of descending air). \remove[HW]{We will also test a}{A} common form of lateral entrainment \cite<e.g.>{DBF+13} \add[HW]{is also tested in section {\protect\ref{sec:results}}}:
\begin{equation}
\varepsilon=\frac{0.2}{r_{c}}
\label{eq:fracEnt}
\end{equation}
where $r_{c}$ is the cloud or plume radius. 
\remove[HW]{Fractional entrainment is related to mass transfers by:}

\subsection{Drag in the Momentum Equation\label{subsec:drag}}

Pressure differences between the fluids can lead to form drag which
is parameterised following \citeA{SW69}, \citeA{RC15} and \citeA{WM19} as: 
\begin{equation}
\mathbf{D}_{ij}=\frac{\sigma_{j}C_{D}}{r_{c}}|\mathbf{u}_{i}-\mathbf{u}_{j}|\left(\mathbf{u}_{i}-\mathbf{u}_{j}\right)\label{eq:dragBubble}
\end{equation}
where $C_{D}$ is a drag coefficient, $r_{c}$ is a cloud radius.
Sensitivity to $C_{D}/r_{c}$is explored in section \ref{sec:results}. 

\subsection{The Buoyancy and the Momentum of the Mass that is Transferred \label{subsec:transferProperties}}

\change[HW]{The different}{Each} fluid may not be well mixed so the fluid transferred
may not have the mean properties of the fluid it is leaving \cite<as was assumed by>{WM19}. \change[WM]{In fact, the most buoyant air should be transferred from fluid 0 to fluid 1 and vice versa, and the air with least downward momentum should be transferred from 0 to 1 and vice versa. The properties of the fluids transferred should depend on the modelling of sub-grid-scale variability which is beyond the scope of this paper.}{A realistic estimate of the properties of the fluid transferred can only be based on knowledge of the sub-filter-scale variability (i.e. the variability within each fluid). Modelling of sub-filter-scale variability is beyond the scope of this paper.}
%
\change[WM]{In section 4 we conditionally average high resolution solutions based on the sign of $w$ so that $w\le 0$ air is in fluid 0 and $w>0$ air in fluid 1. Therefore the air that is transferred at the interface will have $w=0$.}{However, since conditional averaging in this study is based on the vertical velocity, it is possible to say something about the vertical velocity transfer. The subset of fluid 0 that is about to rise (i.e. within the next timestep) should be transferred from fluid 0 to fluid 1, and vice versa for the fluid about to fall in fluid 1. The fluid transferred will therefore have a vertical velocity of $w=0$.}
For \change[WM]{two}{three} dimensional multi-fluid simulations \change[WM]{we}{one} \add[HW]{could} assume that the horizontal
velocity transferred is equal to the mean horizontal velocity of the
fluid transferred from\change[WM]{. Therefore}{:}
\begin{equation}
\mathbf{u}_{ij}^{T}=\begin{pmatrix}u_{i}\\
v_{i}\\
0
\end{pmatrix}.
\end{equation}
\add[WM]{However, only one dimensional multi-fluid simulations are presented in this paper.}

\change[HW]{For}{The} buoyancy \add[HW]{transferred} follows \citeA{TEB19}: \remove[HW]{we use}
\begin{equation}
b_{ij}^{T}=\theta_{b}b_{i}+(1-\theta_{b})b_{j}.
\label{eq:bijT_thetab}
\end{equation}
\change[HW]{and present results}{Results are presented} using $\theta_{b}=\frac{1}{2}$ and $\theta_{b}=1$.
\add[HW]{
$\theta_{b}=1$ is described as an upstream approximation by \protect\citeA{Yano14} and is widely used for entrainment \protect\cite<e.g.>{AK67,AS74}.
}
Note that $\theta_{b}\ne1$ implies that the buoyancy of the fluid
transferred depends on the properties of the receiving fluid which
is not logical and can lead to unbounded values of $b_{i}$ in the
fluid that is losing mass (as demonstrated in section \ref{sec:results}).

\add[HW]{The model $b_{ij}^T=0$ is also tested. The rationale is that if fluid zero contains negatively buoyant air and fluid one contains positively buoyant air then the air that is transferred has zero buoyancy.}

\subsection{\label{subsec:dimAnal}\change[HW]{Dimensional}{Scale}
 Analysis}

\change[HW]{Dimensional}{Scale} analysis can guide our choice of $\gamma$ for the parameterisation of the pressure difference between fluids \change{We make}{using} the following assumptions:
\begin{enumerate}
\item The flow is buoyancy dominated.
\item The flow is close to inviscid.
\item The flow is slowly varying.
\item Fluid $i$ properties are anomalies from a neutrally stable resting
mean state 
\add[HW]{implying that the average over all fluids of $b_i$, $\mathbf{u}_i$ and $P_i$ are zero}.
\item \change[WM]{We consider only the vertical direction}{Only the vertical direction is considered} where $w_{ij}^{T}=0$.
\item $P_{i} = P + \gamma\sum_j \sigma_j\nabla\cdot\mathbf{u}_j 
                              - \gamma\nabla\cdot\mathbf{u}_{i}
             \approx-\gamma\nabla\cdot\mathbf{u}_{i}
            =-\gamma\frac{\partial w_{i}}{\partial z}$
\item \add[HW]{$\frac{\partial \sigma_i}{\partial z} \approx 0$}
\item There are two fluids \remove[HW]{$i$ and $j$}
\item Without loss of generality\add[WM]{,} \change[WM]{we assume that}{the transfers} $S_{ij}=-\frac{\partial w_{i}}{\partial z}>0$
and $S_{ji}=0$ \add[WM]{are used}.
\item \add[Dan]{The drag term, $\mathbf{D}_{ij}$, is zero.}
\end{enumerate}
\change[HW]{This leads to the following balance in the momentum equation}{These assumptions can be used to simplify the multi-fluid $w$ equation, ({\protect\ref{eq:mom})}}:
\begin{equation}
\frac{\partial w_{i}^{2}}{\partial z}-\gamma\frac{\partial^{2}w_{i}}{\partial z^{2}}=b_{i}\label{eq:wi_balances}
\end{equation}
\change[HW]{We}{This is} non-dimensionalised using a length scale $L$, a buoyancy scale
$B$ and a time scale $T$ to get the non-dimensional variables:
\begin{eqnarray*}
\tilde{w} & = & w_{i}T/L\\
\tilde{b} & = & b_{i}/B\\
\tilde{z} & = & z/L
\end{eqnarray*}
\change[WM]{Then the}{The} non-dimensional version of (\ref{eq:wi_balances}) is \add[HW]{then}:
\begin{equation}
\underbrace{{\frac{L}{T^{2}}\frac{\partial\tilde{w}^{2}}{\partial\tilde{z}}}}_{\text{advection}}-\underbrace{{\frac{\gamma}{TL}\frac{\partial^{2}\tilde{w}}{\partial\tilde{z}^{2}}}}_{\text{stabilisation}}=\underbrace{B\tilde{b}}_{\text{buoyancy}}.\label{eq:wi_nonDomTmp}
\end{equation}
The flow is buoyancy dominated so the buoyancy term should be $O(1)$
\change[HW]{so we choose}{which implies} the scaling $B=L/T^{2}\implies T=(L/B)^{\frac{1}{2}}$. \change[HW]{so}{Therefore,} the non-dimensional momentum equation becomes:
\begin{equation}
\underbrace{{\frac{\partial\tilde{w}^{2}}{\partial\tilde{z}}}}_{\text{advection}}-\underbrace{{\frac{\gamma}{B^{\frac{1}{2}}L^{\frac{3}{2}}}\frac{\partial^{2}\tilde{w}}{\partial\tilde{z}^{2}}}}_{\text{stabilisation}}=\underbrace{\tilde{b}}_{\text{buoyancy}}.\label{eq:wi_nonDom-1}
\end{equation}
The stabilisation term must be large enough to smooth out oscillations
in $\tilde{w}$ due to advection and buoyancy but not too large to
remove all variability in $\tilde{w}$. \change[HW]{so we need}{Therefore the stabilisation term should be $O(1)$ which implies that}:
\begin{equation}
\gamma\sim B^{\frac{1}{2}}L^{\frac{3}{2}}.\label{eq:gammaDimAnal}
\end{equation}
\add[HW]{This will be used to guide the choice of $\gamma$ in section {\protect\ref{sec:results}}.}


\section{\label{sec:numerics}Numerical \change[HW]{Solution}{Method}}

The spatial discretisation follows exactly \citeA{WM19}, solving
advective form equations using monotonic, finite volume advection
and C-grid, Lorenz staggering. The time-stepping is Crank-Nicolson
with no off-centering and with deferred correction of explicitly solved
variables. \change[WM]{It differs from}{The method for calculating a separate pressure for each fluid is new because} \citeA{WM19}
\change[WM]{because here we solve the Boussinesq equations and we need stable solutions for equations with a separate pressure in each fluid.}{solved the Euler equations with a single pressure.}
\change[WM]{We use two outer iterations per
time step}{Two outer iterations per timestep are used} to update explicitly solved variables. For the first outer
iteration, predicted values for time $n+1$ are set to those from
time level $n$ and are given the superscript $\ell$. For the second
iteration, values at $\ell$ are set to those from the end of the
first iteration. The implicit numerical method for applying the transfers
between fluids is specific for two fluids because it involves the
inversion of $2\times2$ matrices. 

The prognostic variables are \remove[HW]{the} $b_{i}$ in cell centres, \remove[HW]{the} $\sigma_{i}$
in cell centres and the velocity flux at cell faces, $u_{i}=\mathbf{u}_{i}\cdot\mathbf{S}$,
which is the velocity at cell faces dotted with the area vector for
each face (normal to the face). The pressures, $P$ and $p_{i}$ are
diagnostic \add[Dan]{and are located at cell centres}.

\change[HW]{The first equation to be solved each outer iteration is eqn}{The first calculation of each outer iteration is for the transfer terms, $S_{ij}$ using ({\protect\ref{eq:Sdiv}}) or ({\protect\ref{eq:fracEnt}}). Next}
(\ref{eq:sigma}) \add[HW]{is solved}
for each $\sigma_{i}$, operator split; first applying advection,
then correcting $\sum_{i}\sigma_{i}=1$ and \change[HW]{then}{finally} applying mass transfers:
\begin{eqnarray}
\sigma_{i}^{\prime} & = & \sigma_{i}^{n}-\frac{\Delta t}{2}\nabla\cdot\left\{ \left(\mathbf{u}_{i}^{n}+\mathbf{u}_{i}^{\ell}\right)\left(\sigma_{i}^{n}\right)_{vL}\right\}
\\
\sigma_{i}^{\prime\prime} & = & \sigma_{i}^{\prime}\bigg/\sum_{i}\sigma_{i}^{\prime}
\\
\sigma_{i}^{n+1} & = & \sigma_{i}^{\prime\prime}+\Delta t\sum_{j\ne i}\sigma_{j}^{\prime\prime}S_{ji}-\sigma_{i}^{\prime\prime}S_{ij}
\end{eqnarray}
where $\Delta t$ is the time step and the superscripts $\prime$
and $\prime\prime$ denote intermediate values. $\sigma_{i}^{n}$
is interpolated from cell centres onto cell faces using monotonic
van Leer advection (operator $(\sigma_{i}^{n})_{vL}$) as in \citeA{WM19}
and values exclusively at time level $n$ for best accuracy and guaranteed
monotonicity. For calculating the divergence, the normal component
of $\mathbf{u}_{i}$ is needed at cell faces which are prognostic
variables of the C-grid. The $S_{ij}$ are limited to ensure $0<\sigma_{i}<1\ \forall i$. 

Next eqn (\ref{eq:b}) is solved for the buoyancy in each fluid, first
transporting buoyancy and then applying the transfer terms.
\add[HW]{The advection term, $\mathbf{u}_i\nabla b_i$ is split into $\nabla\cdot(\mathbf{u}_i b_i) - b_i \nabla\cdot\mathbf{u}_i$ for better conservation as calculated by the finite volume method.}
The transfer
terms can be very large due to the presence of $\frac{\sigma_{j}}{\sigma_{i}}S_{ij}$
in the transfer term for $b_{i}$ so they are treated operator split
and implicitly:
\begin{eqnarray}
b_{i}^{\prime} & = & b_{i}^{n}-\frac{\Delta t}{2}\Biggl(\left(\nabla\cdot\left(\mathbf{u}_{i}(b_{i})_{vL}\right)+b_{i}\nabla\cdot\mathbf{u}_{i}+\alpha\nabla^{2}b_{i}\right)^{n}\label{eq:transportb}\\
 &  & \hfill+\left(\nabla\cdot\left(\mathbf{u}_{i}(b_{i})_{vL}\right)+b_{i}\nabla\cdot\mathbf{u}_{i}+\alpha\nabla^{2}b_{i}\right)^{\ell}\Biggr)\\
b_{0}^{n+1} & = & b_{0}^{\prime}+\Delta t\left\{ \left(\frac{\sigma_{1}}{\sigma_{0}}\right)^{\prime\prime}S_{10}\left(b_{1}^{n+1}+b_{10}^{t}-b_{0}^{n+1}\right)-S_{01}b_{01}^{t}\right\} \label{eq:b0np1}\\
b_{1}^{n+1} & = & b_{1}^{\prime}+\Delta t\left\{ \left(\frac{\sigma_{0}}{\sigma_{1}}\right)^{\prime\prime}S_{01}\left(b_{0}^{n+1}+b_{01}^{t}-b_{1}^{n+1}\right)-S_{10}b_{10}^{t}\right\} \label{eq:b1np1}
\end{eqnarray}
where $b_{ij}^{t}=b_{ij}^{T}-b_{i}$. The use of $(\sigma_{0}/\sigma_{1})^{\prime\prime}$
from before the mass transfers is necessary to ensure bounded and
conservative transfers, as described by \cite{MWH20}. Simultaneous
solutions of (\ref{eq:b0np1}) and (\ref{eq:b1np1}) give:
\begin{eqnarray}
b_{0}^{n+1} & = & \frac{\left(1+T_{01}\right)\left(b_{0}^{\prime}-\Delta tS_{01}b_{01}^{t}+T_{10}b_{10}^{t}\right)+T_{10}\left(b_{1}^{\prime}+T_{01}b_{01}^{t}-\Delta tS_{10}b_{10}^{t}\right)}{1+T_{01}+T_{10}}\\
b_{1}^{n+1} & = & \frac{b_{1}^{\prime}+T_{01}b_{0}^{n+1}+T_{01}b_{01}^{t}-\Delta tS_{10}b_{10}^{t}}{1+T_{01}}
\end{eqnarray}
where $T_{ij}=\Delta t\left(\frac{\sigma_{i}}{\sigma_{j}}\right)^{\prime\prime}S_{ij}$

Next a Poisson equation is solved to calculate the total pressure,
$P$ to ensure that the total velocity is divergence free. \add[HW]{This is the standard approach for ensuring divergence constraints \protect\cite<e.g.>{SKW14}.} The pressure
equation also provides updates for the velocity flux, $u_{i}$. First
an intermediate velocity flux is calculated with partial updates and
updates from a previous value of $p_{i}$ \add[HW]{but without updates from $\nabla P^{n+1}$}:
\begin{eqnarray}
u_{i}^{\prime}= & u_{i}^{n}+\frac{\Delta t}{2} & \biggl(\left(-\nabla\cdot(\mathbf{u}_{i}\mathbf{u}_{i})+\mathbf{u}_{i}\nabla\cdot\mathbf{u}_{i}+b_{i}\mathbf{k}\right)_{f}^{\ell}\cdot\mathbf{S}-\nabla_{f}p_{i}^{\ell}\\
 &  & +\left(-\nabla\cdot(\mathbf{u}_{i}\mathbf{u}_{i})+\mathbf{u}_{i}\nabla\cdot\mathbf{u}_{i}+b_{i}\mathbf{k}\right)_{f}^{n}\cdot\mathbf{S}-\nabla_{f}(P+p_{i})^{n}\biggr)
\end{eqnarray}
where operator $()_{f}$ means linear interpolation from cell centres
onto cell faces and $\nabla_{f}P$ is a compact discretisation of
$(\nabla P)_{f}\cdot\mathbf{S}$: i.e. the normal component of the pressure
gradient calculated from just the values of pressure either side of
the face. Once \change[WM]{we have calculated $P^{n+1}$}{$P^{n+1}$ is calculated}, the velocity flux (without
momentum transfers) is updated from the back-substitution:
\begin{equation}
u_{i}^{\prime\prime}=u_{i}^{\prime}-\frac{\Delta t}{2}\nabla_{f}P^{n+1}.\label{eq:backSub}
\end{equation}
The Poisson equation for $P^{n+1}$ is found by multiplying each eqn
(\ref{eq:backSub}) by $\sigma_{i}$, summing over all fluids and
taking the divergence. Knowing that $\nabla\cdot\sum_{i}\sigma_{i}\mathbf{u}_{i}=0$,
$\sum_{i}\sigma_{i}=1$ and $\sum_{i}\sigma_{i}p_{i}=0$ gives:
\begin{equation}
\nabla\cdot\sum_{i}\sigma_{i}u_{i}^{\prime}-\frac{\Delta t}{2}\nabla^{2}P^{n+1}=0
\end{equation}
which is solved to find $P^{n+1}$. 

Next, the momentum transfers due to mass transfer and due to drag
are calculated implicitly, similar to the implicit transfers of buoyancy,
calculating $u_{i}^{n+1}$ from $u_{i}^{\prime\prime}$. These do
not influence the total $P$ or total divergence because momentum
is transferred conservatively. However they do influence $p_{i}$
which is why they are calculated before $p_{i}$. 
\change[Dan]{The final stage of the time step iteration}{The next step}
is to solve Helmholtz equations implicitly for each $p_{i}^{n+1}$:
\begin{equation}
p_{i}^{n+1} = \gamma \sum_j \left(\sigma^{n+1}_j \nabla\cdot u_j^{n+1} \right)
-\gamma\nabla\cdot
\left(u_{i}^{n+1}-\frac{\Delta t}{2}\nabla_{f}\left(p_{i}^{n+1}-p_{i}^{\ell}\right)\right)
\label{eq:pi_Helm}
\end{equation}
Both $p_{i}^{n+1}$ and $p_{i}^{\ell}$ appear in eqn (\ref{eq:pi_Helm})
because $u_{i}^{n+1}$ already contains contributions from $p_{i}^{\ell}$
so these are removed before updating $p_{i}^{n+1}$. No back substitution
for $u_{i}$ is done after calculating $p_{i}^{n+1}$ because this
would destroy the divergence free constraint. Therefore $p_{i}^{n+1}$
only satisfies eqn (\ref{eq:Pi_div}) exactly if the outer iterations
converge, \change[HW]{and}{when} $p_{i}^{\ell}=p_{i}^{n+1}$. 

The final stage of the time step iteration is to reconstruct $\mathbf{u}_{i}$
from $u_{i}$ following \citeA{WM19}. The stages calculating $P$,
$p_{i}$ and $\mathbf{u}_{i}$ are repeated twice per time step iteration.
The Poisson and Helmholtz equations are solved to a tolerance of $10^{-6}$
\add[HW]{using the OpenFOAM conjugate gradient solver with incomplete Cholesky preconditioning}.

\subsection{\label{sec:cAveraging}\add[HW]{Discrete Conditional Averaging} }

\note[HW]{This whole sub-section is new}

The multi-fluid equations will be evaluated in section \ref{sec:results} by comparison with high resolution single fluid solutions. This has to be done carefully as even these reference solutions exist on a discrete grid. A single-fluid solution is conditionally averaged by:
\begin{equation}
i =
\begin{cases}
    0 & \text{ if }w\le 0 \\
    1 & \text{ otherwise.}
\end{cases}
\end{equation}
If an entire grid box is assigned to be either in fluid 0 or fluid 1 based on $w$ at the cell centre then $\sigma$ can only take a small number of discrete values depending on the resolution of the conditional averaging. For example, if conditional averaging goes from $n$ columns to 1 column with vertical resolution staying the same, then $\sigma$ can only take values $\frac{k}{n}$ where $k$ is an integer between zero and $n$. This means that $\sigma$ is either uniform (for small $n$) or has jumps in the vertical (for larger $n$). An alternative is therefore proposed in order to approximate $\sigma$ based on $w$ and $\nabla w$ at cell centres. This is based on the intersection between a grid box and the surface of $w=0$. Given a grid box with centre $\mathbf{x}_c$, the surface of points, $\mathbf{x}_w$ of $w=0$ satisfy:
\begin{equation}
\nabla w \cdot \left(\mathbf{x}_c - \mathbf{x}_w\right) = 0
\end{equation}
Rather than the expense of calculating exact grid box volume fractions either side of this surface, the signed distance of each grid box vertex, $\mathbf{x}_v$, to the surface is calculated:
\begin{equation}
d_v = \left(
     \mathbf{x}_v - \mathbf{x}_c + \frac{w \nabla w}{|\nabla w|^2} 
\right) \cdot \frac{\nabla w}{|\nabla w|}.
\end{equation}
The volume fraction either side of the $w=0$ surface is approximated by the average distance to vertices in one side of the surface divided by the average distance to vertices on both side:
\begin{equation}
\sigma_0 = 
\frac{-\text{mean}\bigl(\text{all } d_v \text{ with } d_v \le 0\bigr)}
{\text{mean}\bigl(\text{all } |d_v| \bigl)}
\;\;\;\;\;\;
\sigma_1 = 
\frac{\text{mean}\bigl(\text{all } d_v \text{ with } d_v > 0\bigr)}
{\text{mean}\bigl(\text{all } |d_v| \bigl)}.
\end{equation}
This gives smooth $\sigma$ fields that will be shown in section \ref{sec:results}.

\section{\label{sec:results}\change[HW]{Validation}{Evaluation} Test Cases}

Two test cases are developed to tune parameters and to \change[HW]{validate}{evaluate} the
use of \change[HW]{multi}{two}-fluid equations to represent dry, two-dimensional, sub-grid-scale
convection. The first is transient; the standard rising bubble of
\citeA{BF02}. \remove{We aim to reproduce the vertical heat transport, the
mean velocity of the rising air, the bouyancy of the rising air and
pressure differences between two fluids in a single column two fluid
model.} The second \remove[HW]{test case} is steady state with heat transfer at
the ground and uniform radiative cooling \add[HW]{(radiative convective equilibrium)}. For both test cases, the
reference solution is a single-fluid solution of the two-dimensional
Boussinesq equations at high horizontal and vertical resolution. \add[HW]{These will be compared with the two-fluid model run using one column.} \add[WM]{The aim is to reproduce} \add[HW]{the vertical heat transport, the mean velocity of the rising air, the bouyancy of the rising air and the pressure differences between fluids in one column of the two-fluid model.}

\subsection{Rising Bubble}

The \add[HW]{two dimensional} test case of \citeA{BF02} was designed for fully compressible
\add[HW]{inviscid models} but similar solutions are obtained solving \add[Dan]{inviscid} Boussinesq equations \add[HW]{($\nu=0$, $\alpha=0$)}.
The Boussinesq version has no background stratification in a domain
of height 10\,km and width 20\,km initially at rest. The buoyant
perturbation is given by:
\begin{equation}
b=\frac{1}{15}\cos^{2}\left(\frac{\pi L}{2}\right)\label{eq:thetaPerturb}
\end{equation}
for $L<1$ where $L=\sqrt{\left(\frac{x-x_{c}}{x_{r}}\right)^{2}+\left(\frac{z-z_{c}}{z_{r}}\right)^{2}}$,
$x_{c}=10\ \text{km}$, $z_{c}=2\ \text{km}$ and $x_{r}=z_{r}=2\ \text{km}$.
100\,m grid spacing is used in the $x$ and $z$ directions for the
two dimensional reference simulation and $\Delta z=100\ \text{m}$
is used in the vertical for the \change[HW]{single}{one} column simulations. All simulations
use a time-step of 2\,s.

\subsubsection{Single-Fluid Solutions}

\change[HW]{The solution of the rising buoyant bubble solving the Boussinesq equations for 1000s is shown in figure \ref{fig:bubble}}{The solution of the Boussinesq equations in two dimensions for the rising bubble at high resolution is shown in figure \ref{fig:bubble}, at the initial time, 500 seconds and 1000 seconds,}
with buoyancy coloured, pressure contoured and the $w=0$ contour dotted.
The solution is very similar to the fully compressible solution \cite{BF02}.
The pressure gradients accelerate the flow beneath the bubble, decelerate
the flow above the bubble and pull in air horizontally. 
\add[HW]{The $w=0$ contour at the initial time shows the initial tendency of $w$ which demonstrates that there is no straightforward relationship between buoyancy and $w$; $w$ is the solution of an elliptic problem involving the momentum equation and the divergence-free constraint. After 500 seconds the centre of the bubble has risen further than the edges and after 1000 seconds there is a complex region of entrainment inside the bubble consisting of both buoyant and neutral rising air and buoyant and neutral sinking air. It is unlikely that the simple parameterisations proposed here will be able to reproduce statistics of this complex behaviour.}

\begin{figure}
\noindent \begin{centering}
\includegraphics[width=0.9\textwidth]{bubble.pdf}
\par\end{centering}
\caption{\label{fig:bubble}The single-fluid, resolved \add[HW]{2D} rising bubble after \add[HW]{0s, 500s and }1000s. Buoyancy is coloured and pressure is contoured every $10\text{m}^{2}\text{s}^{-2}$,
negative contours dashed. The $w=0$ contour is dotted.}
\end{figure}

\citeA{TEB19} and \citeA{WM19}
\change[HW]{assumed that both fluids would share}{used} the same pressure
\add[HW]{for both fluids} with differences parameterised as drag. 
\add[HW]{The high pressure above and low pressure below the bubble in figure {\protect\ref{fig:bubble}} are a manifestation of form drag and will act to slow the ascent. However the pressure patterns in figure {\protect\ref{fig:bubble}} also show the limitations of this approach; the low pressure region at the bottom of the bubble will also act to accelerate the rising flow into the bubble which is not accounted for by drag and there are also widespread pressure anomalies in the sinking fluid which will accelerate the descent whereas drag will always act to slow the fluid.}
\remove[HW]{The pressure in figure {\protect\ref{fig:bubble}} shows that this is a bad assumption since the large pressure anomalies are in the rising fluid. Drag would always act to slow down the fluids whereas the negative pressure gradients in the rising fluid just below the bubble are acting to accelerate the rising motion.}

\change[HW]{Horizontally and conditionally averaged fields, conditioned on $w$, are}{Large scale models of the atmosphere often have coarse horizontal resolution and relatively fine vertical resolution. The high resolution solution is horizontally averaged, conditioned based on the sign of $w$, to give the vertical profiles that coarse resolution models or parameterisations should reproduce. Sinking air and air with $w=0$ is assigned to fluid 0 and rising air to fluid 1 and an average is taken over the $x$ direction at time $t=1000\ \text{s}$ to give the profiles}
shown in the top row of figure \ref{fig:bubble_singleUnderRes}.
$\sigma_{1}$ is the volume fraction of rising fluid at every height and $b_{0}$ and $b_{1}$, $w_{0}$ and $w_{1}$ and $P_{0}$ and $P_{1}$
are the average buoyancy, vertical velocity and pressure of the rising and falling fluids. 
\add[HW]{$\sigma_1$ varies little with height. The complex shape of the bubble is evident in $b_0$ which has two peaks; the lower peak associated with re-entrained air. The fastest ascent is at the bottom of the bubble, co-located with a large drop in pressure. }
\remove[HW]{An accurate multi-fluid model of convection should
reproduce these fields approximately at coarser horizontal resolution
or in a single column model.}

\begin{figure}
\noindent \begin{centering}
\includegraphics[width=\textwidth]{bubble_singleUnderRes.pdf}
\par\end{centering}
\caption{The rising bubble after 1000s simulated with varying horizontal resolution
and a single-fluid. Horizontal averages are conditioned based on $w$.
\label{fig:bubble_singleUnderRes}}
\end{figure}

\add[HW]{In order to see what happens in a single fluid atmospheric model at modest horizontal resolution without a parameterisation of convection, the bubble is simulated at coarser horizontal resolutions, using nine, five, three and one columns instead of the 200 columns of the fine resolution to cover the 20km wide domain and a vertical space of $\Delta z=100\ \text{m}$.}
The \add[HW]{conditional and} horizontal averages \add[HW]{of these simulations} are shown in figure \ref{fig:bubble_singleUnderRes}
\remove[HW]{for horizontal resolutions of nine columns, five columns, three columns and a single column all for the 20km wide domain and all still using
$\Delta z=100\ \text{m}$.}
These simulations are initialised by conservative
horizontal averaging from the full resolution solution. As the horizontal
resolution is reduced, the vertical transport reduces until there
is no movement at all for \change[HW]{a single column}{the one column simulation}: there is no possibility
of fluid rising because in \change[HW]{a single}{one} column with a single-fluid there can be no compensating subsidence. The lack of vertical transport
at coarse resolution motivates convection parameterisation. 

\subsubsection{Results of the Two-fluid, \change[HW]{Single}{One} Column Simulations}

\change[HW]{Multi-fluid, single}{Two-fluid, one} column simulations are initialised by horizontally
and conditionally averaging the high resolution single-fluid initial
conditions. Sinking air is put into fluid zero and rising air into
fluid one. The velocity field after one time step is used for initialisation
as the air is stationary at $t=0$. Results after 500 seconds are shown in figures \ref{fig:multiFluidBubble_500_1}
and \ref{fig:multiFluidBubble_500_2} for various assumptions about
transfers between fluids and various parameter values.

\begin{figure}
\noindent
\includegraphics[width=\linewidth]{multiFluidBubble_500_1.pdf}
\caption{\label{fig:multiFluidBubble_500_1} Results of the rising bubble at $t=500\ \text{s}$. Solid lines are the multi-fluid, \change[HW]{single}{one} column solutions of the rising bubble. Dashed lines are the single-fluid, resolved solutions.}
\end{figure}

\change[HW]{If there is}{The solutions in the top row of figure {\protect\ref{fig:multiFluidBubble_500_1}} have}
no drag between fluids, almost no transfers, no diffusion and
the pressures of both fluids are assumed equal \add[HW]{meaning that} the multi-fluid equations \change[HW]{are}{should be} unstable \cite{TEB19}. 
\change[HW]{The simulation with $S_{ij}=0$ in the top row of figure \ref{fig:multiFluidBubble_500_1} actually has small transfers }{Small transfers are applied}
to prevent either $\sigma_{i}$ from becoming negative. This is sufficient to stabilise the solution but does not prevent oscillations from developing in all fields. The buoyant fluid has risen little from the initial conditions. $\sigma_1$ oscillates between zero and one meaning that only one fluid is present at some locations so there is no possibility for the fluids to move past each other.
\add[HW]{The two fluids have mixed due to the two-way stabilising transfers which have reduced the buoyancy in the rising fluid and increased the buoyancy in the sinking fluid.}

Including entrainment and detrainment as $S_{ij}=-\nabla\cdot\mathbf{u}_{i}$ stabilises the solution and
\add[HW]{leads to a realistic rising volume fraction} (second row of
figure \ref{fig:multiFluidBubble_500_1}).
The fluids move \add[HW]{a little further} past each other
even without pressure differences between the fluids. However \change[HW]{too much is detrained from fluid 1, fluid 1 loses too much buoyancy}{there is too much mixing between the fluids}
and a discontinuity arises at the bubble leading edge; the buoyancy force is producing vertical motion but there is no pressure to make it smooth. 

Using entrainment and detrainment set by divergence ($S_{ij}=-\nabla\cdot\mathbf{u}_{i}$)
and pressure difference between the fluids controlled by divergence
($P_{i}=P+\gamma\sum_j \sigma_j \nabla\cdot\mathbf{u}_j-\gamma\nabla\cdot\mathbf{u}_{i}$), \add[HW]{the} simulations are more realistic (bottom three rows of figure \ref{fig:multiFluidBubble_500_1}).
The first value, $\gamma=2.3\times10^{4}\text{m}^{2}\text{s}^{-1}$, is set from equation
(\ref{eq:gammaDimAnal}) using length scale equal to the initial bubble
radius and a buoyancy scale equal to the initial maximum buoyancy. This makes
the velocity profile too smooth and the updraft too weak so the
bubble does not rise enough. It is reasonable to consider smaller values of $\gamma$ because the buoyancy does not retain its initial maximum for very long. Using $\gamma=10^{4}\text{m}^{2}\text{s}^{-1}$ gives more accurate solutions whereas using $\gamma=4\times10^{3}\text{m}^{2}\text{s}^{-1}$ leads to the leading edge of the bubble rising too quickly \add[HW]{and does not fully remove the discontinuity in the updraft velocity at the leading edge of the bubble. The pressure differences between the fluids have some similarity with the resolved simulation but none of the values of $\gamma$ reproduce the pressure dip at the trailing edge of the bubble. $\gamma=10^{4}\text{m}^{2}\text{s}^{-1}$ gives the closest approximation to the resolved simulation.}
\remove[HW]{peak updraft velocities but $\gamma=10^{4}\text{m}^{2}\text{s}^{-1}$ comes closest, meaning that the bubble rise and $\sigma_{1}$ are approximately correct. The large low pressure at the bottom of the bubble is not reproduced. This is located at the lobes of the highest buoyancy in the resolved simulation (figure {\protect\ref{fig:bubble}}). It would be tempting to model this by assuming hydrostatic pressure or the Bernouilli equation, but these are both forms of the momentum equation which is used to calculate the velocity and so cannot also be used to model the pressure.}

The results in the top row of figure \ref{fig:multiFluidBubble_500_2}
use
\change[HW]{$\gamma=-10^{4}$ and $S_{ij}=-\nabla\cdot\mathbf{u}_{i}$ }{divergence transfer and pressure differences between fluids but, different from previous simulations, also use}
\remove[HW]{but this time with}
drag set by $C_{D}/r_{c}=1/2000\text{m}^{-1}$. The
drag slows the updrafts without increasing the mixing between the
fluids which makes the solutions less like the resolved simulations.
Although drag is known to be an important term in the momentum equations for buoyant thermals and plumes \cite<e.g.,>{RC15}, the drag is related to
\add[HW]{pressure differences caused by the relative motion and to}
mixing with downdraft air.
\change[HW]{which is accounted for in the entrainment term of the momentum equation.}{The mixing with downdraft air is already accounted for in the entrainment and the pressure differences are already accounted for in the pressure parameterisation. Therefore the drag parameterisation is not a useful contribution to the multi-fluid Boussinesq equations.}

\begin{figure}
\noindent
\includegraphics[width=\linewidth]{multiFluidBubble_500_2.pdf}
\caption{As figure \ref{fig:multiFluidBubble_500_1} but different parameterisation options.
\label{fig:multiFluidBubble_500_2}}
\end{figure}

If
\change[WM]{we include pressure difference between fluids}{pressure differences between fluids are included}
with $\gamma=10^{4}\text{m}^{2}\text{s}^{-1}$
\change[WM]{but omit}{without}
entrainment ($S_{ij}\approx 0$) (second row of figure \ref{fig:multiFluidBubble_500_2})
the fluids move past each other but there is 
\change[HW]{an unrealistic discontinuity with no}{a build up of fluid 1 at the leading edge of the bubble and very little}
fluid 1 below the \change[HW]{buoyant fluid}{bubble.}
\remove[HW]{and very little falling air at the lowest height of the rising fluid.}
With \change[HW]{single}{one} column and
two fluids, all air must rise in one fluid and sink in the other.
Continuity does not allow for any other solution. Pressure gradients
can accelerate or decelerate this flow but as the model is set up,
with no transfers between fluids, no fluid can change direction. It
is therefore essential to have transfers between fluids.

The other entrainment option is the commonly used fractional entrainment
rate, $\varepsilon=0.2/r_{c}$. This is used \change[HW]{combined}{in combination with} with pressure differences between fluids in the third row of figure \ref{fig:multiFluidBubble_500_2}.
\change[HW]{This combination produces smooth solutions}{The solutions are smooth}
but the \add[HW]{rising volume fraction is not right with a build up of fluid 1 at the bubble leading edge, the} updraft is too
weak and too much is entrained into fluid 1 \add[HW]{above the bubble}.
Using this entrainment, the updraft will entrain more and more air until $\sigma_{1}=1$. The $S_{ij}=-\nabla\cdot\mathbf{u}_{i}$ model gives
\change[HW]{much more accurate entrainment and detrainment for this test case using conditional averaging based on $w$.}{results closer to the reference solution.}

Two alternative models for estimating the buoyancy of the fluid transferred
between updraft and downdraft have been tested. The results in
the fourth row \change[HW]{if}{of} figure \ref{fig:multiFluidBubble_500_2} assume that the fluid transferred has zero buoyancy in order to keep the
positively buoyant air in fluid one and the negatively buoyant air
in fluid 0. Thus when zero buoyancy air leaves fluid one (the updraft),
the mean buoyancy of fluid one increases. This leads to
\change[HW]{very high buoyancy in fluid one which is not present in the resolved simulation.}{higher buoyancy in fluid one than when using $b_{ij}^T=b_i$ and gives results close to the reference solution.}
The results in the bottom row of figure \ref{fig:multiFluidBubble_500_2}
use $b_{ij}^{T}=\frac{1}{2}\left(b_{0}+b_{1}\right)$ following \citeA{TEB19}
so that the fluid transferred is influenced by the fluid it is moving
towards as well as where it has come from. 
These solutions are \add[HW]{also} similar to the resolved solutions.
\remove[HW]{but the bubble has moved a little too fast.}
\remove[HW]{These simple models reinforce the idea that the buoyancy of the fluid transferred should depend on knowledge of sub-grid-scale variability of $b$ within each fluid. These simple assumptions can lead to unrealistic behaviour. The best model tested is $b_{ij}^{T}=b_{i}$ which assumes that the fluid transferred takes the mean properties with it.}

\remove[HW]{This section has demonstrated the value of using single column, multi-fluid model to simulate a buoyant rising bubble with a single updraft fluid and a single downdraft fluid. Transfers between the fluids have been successfully modelled using entrainment set by divergence giving realistic updraft fractions. Entrainment calculated as the reciprocal of plume width was not successful. The pressure difference between the fluids has also been represented by divergence with a coefficient close to that expected from dimensional analysis. Including drag between the fluids did not improve simulations. Two models of transferring non-mean properties between fluids were tested but neither was plausible.}

\begin{figure}
\noindent
\includegraphics[width=\linewidth]{multiFluidBubble_1000.pdf}
\caption{\label{fig:multiFluidBubble_1000} Results of the rising bubble at $t=1000\ \text{s}$. Solid lines are the multi-fluid, \change[HW]{single}{one} column solutions of the rising bubble. Dashed lines are the single-fluid, resolved solutions.
}
\end{figure}

\add[HW]{
From the results at 500 seconds, we can conclude that:
\protect\begin{itemize}
\item Divergence transfer (entrainment and detrainment) leads to realistic updraft area fractions.
\item Pressure differences between fluids are usefully modelled by divergence, with a coefficient close to the derived value from scale analysis.
\item Drag terms are not useful.
\item Entrainment calculated as the reciprocal of plume width is not useful.
\protect\end{itemize}
In order to examine further the effects of different assumptions about the buoyancy of the fluid transferred, a selection of results at 1000 seconds are presented in figure {\protect\ref{fig:multiFluidBubble_1000}}. 
None of the simulations reproduce the complex buoyancy pattern at the trailing edge of the bubble associated with a deep low pressure. The rising air fraction ($\sigma_1$) at and below the bubble is too low in all simulations. The top row which uses $b_{ij}^T=b_i$ (transferring the mean buoyancy) shows too much mixing between the fluids -- too much of the most buoyant air has been transferred out of fluid 1 but the leading edge of the bubble has risen about the right amount. In the other two simulations, the air that is transferred out of fluid 1 has lower buoyancy than the mean in fluid 1 ($b_{10}^T<b_1$) so the remaining rising fluid stays very buoyant and rises too far. The assumption that the buoyancy transferred is zero (middle row of figure {\protect\ref{fig:multiFluidBubble_1000}}) means that the buoyant fluid increases its buoyancy above the initial conditions which would not be possible in the resolved single-fluid solution.}

\add[HW]{It might be possible to improve aspects of this simulation by optimising $\gamma$ and $\theta_b$ (the coefficient from eqn ({\protect\ref{eq:bijT_thetab}}) for calculating $b_{ij}^T$) but it is probably not helpful to over fit the two-fluid model to this dry, two dimensional test case.
There are clearly discrepancies between
the single-fluid resolved rising bubble and the two fluid one column rising bubble but rather than trying to fit this test case closely, cases with multiple plumes in each column should be considered. Using the lessons learned from modelling the rising bubble, a statistically stationary case with many plumes is simulated next.}

\subsection{Radiative-Convective Equilibrium (RCE)}

A two-dimensional, dry radiative convective equilibrium test case
is devised to mimic some properties of atmospheric convection, in
particular the difficulty resolving the flow at coarse resolution.
The domain is 160km wide, 10km tall and is resolved by 640 cells in
the horizontal and 40 in the vertical ($\Delta x=\Delta z=250$m).
The top and bottom boundaries are zero velocity and the lateral boundaries
are periodic. A heat flux of $h=10^{-3}\text{m}^{2}\text{s}^{-3}$
is imposed at the bottom boundary leading to a boundary condition
of $\partial b/\partial z=-h/\alpha\text{s}^{-2}$ where $\alpha=100\ \text{m}^{2}\text{s}^{-1}$
is the buoyancy diffusivity. The top boundary has $\partial b/\partial z=0$.
Diffusion is applied to the momentum equation with a coefficient $\nu=70.7\ \text{m}^{2}\text{s}^{-1}$
so as to give a Prandtl number of $0.707$, the value for dry air. Uniform cooling of $-1\times10^{-7}\text{m}\text{s}^{-3}$ is applied to the domain to maintain equilibrium.

The fluid is initially stationary with $b=0$. In order to quickly
and reproducibly initialise instability, $b=10^{-4}\text{m}^{2}\text{s}^{-1}$
is imposed in a square of side length 2000m at the centre in the $x$
direction and at the ground. The simulation is run using a time step
of $5\ \text{s}$. A quasi-steady, choatic state is reached after
about $3\times10^{4}\text{s}$ and the simulation is run for $2\times10^{5}\text{s}$
and conditional averages are calculated over the final $10^{5}$s.

This test case is designed to have strong updrafts in narrow plumes
and weak descent elsewhere to mimic atmospheric convection but without
the complication of moisture, phase changes and three spatial dimensions.

\subsubsection{Single-Fluid Solutions}

A snapshot of the buoyancy, vertical velocity and pressure of the
radiative-convective equilibrium test case at $2\times10^{5}\text{s}$
are shown in figure \ref{fig:RCE_resolved} showing intense, narrow
updrafts. The buoyant air at the ground rises in narrow plumes,
accelerating due to buoyancy and pressure gradients then decelerating
due to pressure gradients before reaching the top. The plumes spread
out and the sinking air is a mixture of warm and cold with similar
pressure gradients in the rising and falling air, accelerating then
decelerating the flow so as to maintain continuity. 

\begin{figure}
\noindent 
\includegraphics[width=1\textwidth]{RCE_resolved.pdf}
\caption{\label{fig:RCE_resolved}
Buoyancy, vertical velocity and pressure at $t=2\times10^{5}\ \text{s}$
of the resolved radiative-convective equilibrium test case. Vertical
velocity contours are black every 0.5m/s with negative contours dashed
and the zero contour dotted. The domain is 10km high and 160km wide
and the z-direction is stretched a factor of 2 in the plots.}
\end{figure}

Horizontal averages conditioned on $w$ and time averaged every 1000s
between $10^{5}\ \text{s}$ and $2\times10^{5}\ \text{s}$ are shown
\change[HW]{in the top row of}{as dashed lines in} figure \ref{fig:RCE_singleColumn}. The rising and
falling volume fractions, $\sigma_{1}$ and $\sigma_{0}$, remain
close to $\frac{1}{2}$ throughout the depth. The rising plumes look
narrow in figure \ref{fig:RCE_resolved} but there are wide regions
of slowly rising air around the plume cores. The entrained rising
cool air and warm air sinking that has hit the model top mean that
the rising and falling air have similar average buoyancy away from
the ground. The ascending air gradually accelerates towards the middle
of the domain then decelerates towards the model top. At the ground
the acceleration is forced by buoyancy and away from the ground the
acceleration is controlled by pressure gradients. The descending air
mean velocity is close to equal and opposite to the ascending air,
as expected since $\sigma_{1}=0.5$.

\subsubsection{Results of the Two-fluid, \change[HW]{Single}{One} Column Simulations}

The two fluid model is used to simulate the RCE test case in \change[HW]{single}{one}
column of 40 cells with zero buoyancy gradient at the top and a buoyancy
gradient of $10^{-5}\text{s}^{-2}$ at the ground for both fluids
and uniform radiative cooling. The two fluids are initialised using
the time mean conditional horizontal averages from the single-fluid
resolved simulation. The entrainment model $S_{ij}=-\nabla\cdot\mathbf{u}_{i}$
is used for all simulations to account for transfers throughout the
domain to represent cloud base mass flux (at the bottom boundary),
cloud top detrainment and lateral entrainment. $\gamma$ is set from
equation (\ref{eq:gammaDimAnal}) using the depth of the \add[HW]{high resolution solution} super adiabatic
layer as the length scale ($L=800\text{m}$) and the buoyancy scale
$B=L\ \partial b/\partial z=8\times10^{-3}\text{m}\text{s}^{-2}$
giving $\gamma=2000\text{m}^{2}\text{s}^{-1}$. \add[HW]{(Scale analysis or boundary layer parameterisation could instead be used to estimate $L$ independently.)}
A simulation using
$S_{ij}=-\nabla\cdot\mathbf{u}_{i}$, no drag and $b_{ij}^{T}=b_{i}$
is shown in the top row of figure \ref{fig:RCE_singleColumn}.
The entrainment model $S_{ij}=-\nabla\cdot\mathbf{u}_{i}$ keeps $\sigma_{i}$
at about 0.5 throughout the depth, as in the resolved simulation,
with entrainment in the lower half of the domain ($S_{01}>0$) and
detrainment in the upper half ($S_{10}>0$), where the air decelerates
towards the top. The \change[HW]{single}{one} column model reproduces the super-adiabatic
layer near the ground and the near uniform buoyancy away from the
ground but the buoyancy difference between the fluids is too large.
The updraft and downdraft velocities are captured well. The pressure
differences between fluids are too large which is consistent with
the over estimated buoyancy difference between fluids. 

\begin{figure}
\noindent
\includegraphics[width=\linewidth]{RCE_singleColumn.pdf}
\caption{\label{fig:RCE_singleColumn}
Results of the RCE simulations. \protect\change[HW]{The top line shows}{Dashed lines show} horizontal averages
of the resolved case between $t=10^{5}\text{s}$ and $2\times10^{5}\text{s}$. \protect\change[HW]{The bottom two rows show}{Solid lines show} results of the \change[HW]{single}{one} column multi-fluid model after $10^{4}\ \text{s}$. }
\end{figure}

Results of a simulation\remove[WM]{s} using $b_{ij}^{T}=\frac{1}{2}(b_{0}+b_{1})$
is shown in the bottom row of figure \ref{fig:RCE_singleColumn}.
This model of the buoyancy of the transferred fluid is \change[HW]{bad}{not helpful}; it leads
to strong cooling in fluid zero near the ground due to the large transfers.
It is more accurate to assume that fluids take their mean properties
when transferred, $b_{ij}^{T}=b_{i}$. 

\add[HW]{The two-fluid, one column model of RCE gave results closer to the reference solution than for the rising bubble. This is likely to be because} \add[WM]{the rising and falling fluids in the RCE test case} \add[HW]{are averaged over a large area with plumes at different stages of development rather than trying to simulate the complex entrainment pattern behind a single bubble. The RCE test case is horizontally homogeneous and in equilibrium making it more straightforward.}

\section{Summary, Conclusions and Further Work \label{sec:concs}}

\add[HW]{This paper demonstrates the plausible utility of using multi-fluid equations to represent sub-grid-scale convection.}
\change[HW]{This paper presents solutions of the multi-fluid Boussinesq equations for convection, reproducing aspects of highly resolved dry convection
at coarse horizontal resolution.}{Solutions of the two-fluid Boussinesq equations using coarse horizontal resolution (one column) reproduce aspects of highly resolved single-fluid simulations of dry, two-dimensional convection.}
Two test cases are used to \change[HW]{validate}{evaluate}
the multi-fluid model. The first is a Boussinesq version of the well
known two dimensional dry rising bubble of \citeA{BF02} and the second
is a dry two dimensional version of radiative-convective equilibrium
with heating at the ground and uniform radiative cooling. High resolution
single-fluid simulations are conditionally averaged based on the sign
of vertical velocity ($w$) in order to \change[HW]{create}{define} rising and falling fluids which each have their own \remove[HW]{mean} buoyancy, velocity and pressure. The two-fluid, \change[HW]{single}{one} column Boussinesq model reproduces mean vertical heat and mass transport of these test cases. 

\add[HW]{The radiative-convective equilibrium (RCE) test case complies with the assumptions of a traditional mass-flux convection scheme --  an ensemble of plumes per column, horizontal homogeneity and equilibrium. The rising bubble test does not satisfy these constraints. The multi-fluid model gives good solutions for the RCE test case and good solutions for the bubble for the first part of the simulation but does not represent the complex re-entrainment as the bubble continues to rise.}

The multi-fluid Boussinesq equations can have divergence in each fluid
while the divergence averaged over all fluids is 
\change[HW]{zero}{constrained to be zero by the single-fluid Boussinesq equations.}
A model is proposed and \change[HW]{validated}{evaluated} for the pressure difference between the rising
and falling fluids based on the divergence of each fluid. The use
of a different pressure for each fluid is not only realistic but also
leads to stable solutions of the multi-fluid equations. 

Conditionally averaging the high resolution solutions based on $w$
leads to
\change[HW]{convecting and non-convecting regions}{rising and falling fluids} with area fractions, $\sigma_{i}$, which are approximately uniform with height. Models for entrainment and detrainment into and out of \change[HW]{convecting}{these} regions
\change[HW]{must clearly}{need to} be consistent with the definition of the convecting region.
\change[HW]{We propose a}{A} model for entrainment and detrainment based on divergence \add[HW]{is proposed}
that leads to excellent agreement with $\sigma$ from the high resolution
solutions. \change[HW]{We also test a}{A} more conventional fractional entrainment rate based on the reciprocal of the plume width \add[HW]{is also tested} which leads to unrealistic \change[HW]{convecting}{rising} area fractions. 

\change{A set of unknowns in the multi-fluid equations are the}{The} properties of the fluid that are transferred from one fluid to another \add[HW]{are unknowns in the multi-fluid equations and these require closure}. An obvious
first guess is that fluid takes its mean buoyancy and velocity
with it when it is transferred.
\change[HW]{However sub-grid-scale variability of each fluid means that we expect that the most buoyant fluid will be transferred to the rising fluid or the fluid with the largest vertical velocity will be transferred to the rising fluid and vice versa. An accurate estimate of these would depend on knowledge of the sub-grid-scale variability of each fluid which we do not have. Instead we assume that the fluid transferred in each direction has zero vertical velocity
which is consistent with our conditional averaging.}{Improvements on this assumption will depend on knowledge of sub-filter-scale variablity. However since conditional averaging is based on the sign of $w$, it is consistent to assume that the fluid transferred has $w=0$ ($w_{ij}^T=0$).}
\remove[HW]{We have tested}
Three models for the buoyancy of the fluid transferred 
\add[HW]{are tested}. The most successful
is to assume that the fluid takes its mean buoyancy with it
\add[HW]{($b_{ij}^T=b_i$)}.
Assuming that the buoyancy of the transferred fluid is an average of the buoyancy of both fluids \add[HW]{($b_{ij}^T=\frac{1}{2}(b_i+b_j)$)} can lead to unrealistically large or small, unbounded buoyancies. 
\change[HW]{Similarly, if we assume}{Assuming}
that one fluid has negative buoyancy and the other positive buoyancy and 
\remove[HW]{we transfer}
fluid of zero buoyancy \add[HW]{is transferred ($b_{ij}^T=0$) also leads to} unbounded buoyancy. \remove[WM]{A better model of the buoyancy
of the fluid transferred would rely on more information of the
sub-grid-scale variability.}

\change[HW]{Single}{One} column multi-fluid modelling with moisture has already been demonstrated by \citeA{TEB19} 
\add[HW]{and two dimensional simulations of the multi-fluid Euler equations by \protect\citeA{WM19}}.
Two important next steps are to simulate convection in the grey zone (where it is partially resolved) and to represent sub-\change[HW]{grid}{filter}-scale variability within each fluid by solving multi-fluid prognostic equations for turbulent kinetic energy and other moments of the sub-\change[HW]{grid}{filter} variability. 
\remove[HW]{These can be used to inform}
Cloud base mass flux \add[HW]{could then be estimated based on higher-order moments of the vertical velocity and buoyancy within the environment fluid.}
\remove{(since only the most buoyant fluid will transfer
from the stable environment to the buoyant plume) and inform models
of the properties of the fluid that is transferred.}
This will lead
to a unified parameterisation of turbulence and convection and will
take multi-fluid modelling of convection beyond existing paradigms. 

\acknowledgments

Many thanks to Peter Clark, John Thuburn and Chris Holloway for valuable discussions and comments on the manuscript.
Thanks to the NERC/Met Office Paracon project. We acknowledge funding
from the Circle-A and RevCon Paracon projects NE/N013743/1 and NE/N013735/1. The code is built using \url{https://openfoam.org/version/7/} and is part of \url{https://github.com/AtmosFOAM/}. The code releases are \url{https://doi.org/10.5281/zenodo.3828474} and \url{https://doi.org/10.5281/zenodo.3832130}.

\bibliography{numerics}

\end{document}
