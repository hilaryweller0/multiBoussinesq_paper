%% LyX 2.2.4 created this file.  For more info, see http://www.lyx.org/.
%% Do not edit unless you really know what you are doing.
\documentclass[12pt,english]{article}
\usepackage[T1]{fontenc}
\usepackage[latin9]{inputenc}
\usepackage{geometry}
\geometry{verbose,tmargin=2.5cm,bmargin=2.5cm,lmargin=2.5cm,rmargin=2.5cm}
\usepackage{color}
\usepackage{amsmath}
\usepackage{amssymb}
\usepackage[authoryear]{natbib}

\makeatletter
%%%%%%%%%%%%%%%%%%%%%%%%%%%%%% Textclass specific LaTeX commands.
\newenvironment{lyxlist}[1]
{\begin{list}{}
{\settowidth{\labelwidth}{#1}
 \setlength{\leftmargin}{\labelwidth}
 \addtolength{\leftmargin}{\labelsep}
 \renewcommand{\makelabel}[1]{##1\hfil}}}
{\end{list}}

%%%%%%%%%%%%%%%%%%%%%%%%%%%%%% User specified LaTeX commands.
\usepackage{xcolor}
\renewenvironment{quote}
               {\list{}{\rightmargin\leftmargin}%
                \item\relax\color{blue}}
               {\endlist}

\begin{document}

\title{Responses to Reviewer Comments on\\
2019MS001966\\
Multi-fluids for Representing Sub-filter-scale Convection}

\author{Hilary Weller, William McIntyre, Dan Shipley and Peter Clark}
\maketitle
\begin{quote}
Author responses and additional text in the manuscript are in blue.
\end{quote}

\section*{Responses to Reviewer 1}

The paper is written in a manner that relies too heavily on an assumption
that the reader is familiar with previous works by the authors, yet
these works are rather new in the field of convective parameterizations
and I believe that the authors should give more context. The introduction
could be made more detailed, referencing some key works such as the
Lappen and Randell 2011 (a,b,c) and other approaches to convection
sub as the high order closures and specifically the CLUBB (e.g. Golaz
et al., 2013). Section 2 should place the reader on a better footing
before presenting the model equations (I leave it to the authors to
decide if they need more equations, an appendix, or only text referencing
to their previous works to do so).
\begin{quote}
Response
\end{quote}
The paper present results from a bubble experiment and RCE. The bubble
turns out to be more challenging than the RCE. Can you please elaborate
in your discussion why do you think that is the case? Does it has
anything to do with the fact that in the RCE your multi-fluid updraft
fluid represent the average of many rising plumes while in the bubble
your multi-fluid updraft represents single bubble?
\begin{quote}
Response
\end{quote}

\subsection*{Specific Comments: }
\begin{lyxlist}{00.00.0000}

\item [{Line\ 45:}] ``Improvement have been made \textbf{by}''?
\begin{quote}
 Response
\end{quote}
\item [{Line\ 56}] you wrote ``\cite{TKP+18} didn't
propose a suitable numerical method for solving the equations''.
\cite{TKP+18} do present numerical solutions of their model (BOMEX test
case and some modification of it) - either change that statement or
elaborate what do you mean by ``didn't
propose a suitable numerical method for solving the equations''.
\begin{quote}
 Response
\end{quote}
\end{lyxlist}
\bibliographystyle{abbrvnat}
\bibliography{numerics}

\end{document}
